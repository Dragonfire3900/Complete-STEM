In classical mechanics there are a few fundamental ideal
\begin{enumerate}
    \item Equations of motion are independent of energy "gauge" (reference point of $E$)
    \item Maxwell's equations are independent of the choice of gauge
        $\phi \rightarrow \phi +  \frac{1}{c} \diffp{\Lambda}{t}$, and
        $\vec{A} \rightarrow \vec{A} + \vec{\nabla} \Lambda$. Which can be summarized as
        $A_{\mu} = \left(\phi, -\vec{A}\right) \rightarrow
        A_{\mu} + \partial_{\mu} \Lambda$
\end{enumerate}
Today we are wanting to quantize the motion of a particle moving through a background classical
field and see what the Gauge transformations are.

In quantum mechanics the analogous change of reference energy is modififying the 
Hamiltonian by adding an additional energy term $\mathcal{H} \rightarrow \mathcal{H} + E_0 I$.
From previous sections we know how the wave function evolves under time and this new
transformation. So $\Ket{\psi} \rightarrow \Ket{?}$ under the given $\mathcal{H}$ trans.
Recall
$$
    U_{\mathcal{H}} (t, t_0) \Ket{\psi} =
    e^{-\frac{i}{\hbar} H (t - t_0)}
$$
Since the identity operator commutes with everything we will end up finding that
there is an overall phase shift. Thus,
$$
    \Ket{\psi} \rightarrow e^{-\frac{i}{\hbar} E_0 (t - t_0)} \Ket{\psi}
$$
Thus, in changing the "reference energy we only have changed the phase. Thus,
any operator $A$ is gauge invariant then the expectation over time does not change.

\section{Maxwell Interaction}
Let's start by claiming that the Lagrangian for a QM particle in an electric field is
$$
    \mathcal{L} = \frac{1}{2} m \vec{x}^2 - e \phi + \frac{e}{c} \vec{A} \cdot \vec{x}
$$
Using this we can come up with two different definitions for momentum
\begin{enumerate}
    \item Canonical momentum 
        $\diff{{\mathcal{L}}}{{\dot{\vec{x}}}} = m \dot{\vec{x}} + \frac{e}{c} \vec{A} = \vec{p}$
    \item Mechanical or kinematic momentum
        $\vec{\Pi} \equiv m \dot{\vec{x}} = \vec{p} - \frac{e}{c} \vec{A}$
\end{enumerate}
In order for this claimed Lagrangian to be true, it must match the Lorentz force when we look 
at the overall momentum. Recalling it as the following form
$$
    \vec{F} = e \vec{E} + \frac{e}{c} \dot{\vec{x}} \times \vec{B} =
    -e \vec{\nabla}\phi - \frac{e}{c} \diffp{{\vec{A}}}{t} + 
    \frac{e}{c} \dot{\vec{x}} \times \left(\vec{\nabla} \times \vec{A}\right)
$$
Going through the motions of the Euler-Lagrange equation will show that this is the correct
Lagrangian. We can now use the Legendre transform to get the Hamiltonian as
$$
    \mathcal{H} = \vec{p}\cdot\dot{\vec{x}} - \mathcal{L} =
    \frac{1}{2m} \left(\vec{p} - \frac{e}{c} \vec{A}\right)^2 + e \phi
$$
An important thing to notice is that $\vec{A}$ and $\phi$ are operators since they depend
on the location where you are evaluating. Using the various cannonical commutators we can find
the commutation relationship for the mechanical momentum
$$
\left[\Pi_i, \Pi_j\right] = \frac{i\hbar e}{c} \epsilon_{ijk} B_k
$$
Which gives the following equation of motion
$$
    m \diff[2]{{x^H}}{t} = \diff{\vec{\Pi}}{t} =
    e \left[\vec{E} + \frac{1}{2c}
    \left(\diff\right)\right]
$$

\subsection{Magnetic Field in a ring}
We are going to deal with the magnetic field in a ring of conducting material to see how the
magnetic flux quantizes. To start off let's say that the magnetic field has the component
$\vec{B} = B_0 \hat{z}$. From the definition of $\vec{A}$ we can guess what possible values
of $\vec{A}$ make sense. One such guess which is helpful is
$$
    \vec{A} = - \frac{B_0}{2} y \hat{x} + \frac{B_0}{2} x \hat{y} + 0 \hat{z}
$$
If you check this you'll find that it produces the correct setup. This also corresponds to
{\color{red} Coulomb gauge} or $\vec{\nabla}\cdot \vec{A} = 0$. From this you can also work
with this Gauge in cylindrical coordinates $\vec{A} = \frac{B_0 \rho}{2} \hat{\phi}$.
