\documentclass[12pt]{article}

\usepackage{tikz}

\usepackage{mdframed}
\usepackage{amsmath}
\usepackage{amssymb}
\usepackage{amsfonts}
\usepackage{amsbsy}
\usepackage{amsthm}
\usepackage[nomath]{stix}
\usepackage{bbm}
\usepackage{comment}
\usepackage{braket}
\usepackage{mathtools}
\usepackage{esdiff}

\newcounter{hwProblem}
\newcounter{hwPart}[hwProblem]
\setcounter{hwProblem}{1}

\providetoggle{showPrompts}
\settoggle{showPrompts}{true}

\mdfdefinestyle{hwPromptBox}{linecolor=blue}

\newenvironment{hwProblem}[2][-1]{
    \ifnum #1>0
        \section*{Problem \arabic{hwProblem} ({\it #1 pts.}): #2}
    \else
        \section*{Problem \arabic{hwProblem}: #2}
    \fi
    \stepcounter{hwProblem}

    %Defining what a problem part is
    \newenvironment{hwPart}[2][-1]{
        \stepcounter{hwPart}
        \ifnum##1>0
        \subsubsection*{Part \Alph{hwPart} ({\it ##1 points}): ##2}
        \else
        \subsubsection*{Part \Alph{hwPart}: ##2}
        \fi

    }{}
    
    \iftoggle{showPrompts}{
        \newenvironment{hwPrompt}{
            \begin{mdframed}[style=hwPromptBox]
        }{\end{mdframed}}
    }{
        \excludecomment{hwPrompt}
    }
}{}

\DeclarePairedDelimiter{\abs}{\lvert}{\rvert}
\DeclareMathOperator{\tr}{tr}

\newtheorem{theorem}{Theorem}[section]
\newtheorem{corollary}{Corollary}[theorem]
\newtheorem{lemma}[theorem]{Lemma}

\theoremstyle{definition}
\newtheorem{definition}{Definition}[section]

\includecomment{pbStatement}
\includecomment{longform}

\title{Lecture Day \#18}
\author{Joel Kelsey}
\date{10-27-2025}

\begin{document}
\maketitle

\section{Solutions to a radial potential}
Specifically to the equation
$$
    \left(\frac{\vec{p}^2}{2m} + V(r)\right) \psi = E_n \psi
$$
We ended up finding the following solution last class by trying this \newline
$\psi = R_{nl}(r) Y^{m}_{l}\left(\theta, \phi\right)$ and let
$R_{nl}(r) = \frac{U_{nl}(r)}{r}$. Along with this we found
$u = \eta^{l+1} e^{\eta/2} \omega(\eta)$, where
$\eta = 2 \kappa r$, where $\kappa^2 = -2mE_n / \hbar^2 > 0$
and $\omega_{nl}$ satisfies
$$
    \omega'' + \left(\frac{2l+2}{\eta} - 1\right) \omega' +
    \left[\frac{V}{2E_n} - \frac{\left(l+1\right)}{\eta}\right] \omega
$$
We now want to solve this potential for something that corresponds
to the hydrogen atom. Being $V = -Ze^2 / r = -2 Z e^2 \kappa / \eta$.
This simplifies the DE for $\omega$ into
$$
    \eta \omega'' + \left(2l+1 + 1 -\eta\right) \omega' +
    \left[\frac{-Z e^2 \kappa}{2E} - (l+1)\right] \omega =0
$$
This has a regular and unique solution if $N \in \mathbb{Z}_{\ge 0}$ and
builds up solutions of the Associated Laguere polynomials of
$\omega(\eta) = L_{N}^{2l+1}$. Which can be looked up from various tables.
This overall process comes from a solution to the Strum-Louiville DEs.
We can solve for $E_n$ in terms of $N$ and get the following
$$
    E_n = - \left(\frac{Z e^2 \sqrt{2m}}{2\hbar \left(N + l + 1\right)}\right),
    \; n = N + l + 1
$$
This gives a constraint on the various values of $l$ by pointing out that
$l = 0, \dots, n - 1$. We can also solve for the degeneracy of this setup
by $\sum_{l=0}^{n-1} 2l + 1 = n^2$. We aren't expecting this to just be
$n^2$ and indicates that there must be some kind of symmetry we are not
expecting. We will come back to this in Chapter 4 but there is another
classical symmetry that we didn't take into account which produces this
degeneracy

This is one of the few systems which we can explicitly calculate the spectrum of.
This is quite rare as there are only a few systems which have these explicit spectra.
So it's quite cool overall! We also mostly will use these setups
to produce solutions using perturbation theory.

\section{Tensor Products}
We can construct entaglement using tensory products. We'll be going over how this
can be constructed using this setup. 
\begin{definition}[Cartesian Product]
\label{cartProd}
The cartesian product of two vector spaces $V$ and $W$ are all sets s.t.
$$
    V \times W = \left\{(v, w) | v \in V,\; w\in W\right\}
$$
\end{definition}

We want to obtain a vector space from the cartesian product of two vector spaces. In
order to do this we need to define how the various operations on a normal vector space works.
One way to do this is by using element wise definitions which produces a
{\color{red} product space}. This would be illustrated via the following properties
\begin{enumerate}
    \item $\left(v_1, w_1\right) + \left(v_2, w_2\right) = \left(v_1 + v_2, w_1 + w_2\right)$
    \item $\forall k \in \mathbb{C}$, with
        $k (v, w) = (kv, kw)$.
\end{enumerate}
This builds up the product space.
We can find the dimension of this space to be
$\dim \left(V \times W\right) = \dim(V) + \dim(W)$

With a tensor product we can extend this
by imposing additional algebraic structure. In particular,
we want to impose a bilinearity condition in the first and second arguments.
\begin{enumerate}
    \item $\left(\alpha v_1 + \beta v_2, w\right) \equiv
        \alpha \left(v_1, w\right) + \beta \left(v_2, w\right)$ and similarly
        for the second argument
\end{enumerate}
Interestingly if have the second argument being zero then the overall vector
must be multiplied by zero and thus, be the zero vector. We will denote
spaces which have this property as
$(v, w) = v \otimes w$.

With this additional structure the overall dimensionality will be multiplicative instead
of additive. This is because,
if $\left\{e_i\right\}$ and $\left\{f_i\right\}$ are bases of $V_1$ and $V_2$, then
$\left\{e_i \otimes f_i\right\}$ is the basis of $V_1 \otimes V_2$.
This is also where the kronecker product defines the tensor product. Explicitly
any vector can be written as
$\left(\sum a_i e_i\right) \otimes \left(\sum b_j f_j\right) =
\sum_{i,j} a_i b_j \left(e_i \otimes f_j\right)$

In theory we've actually already seen this type of space! We actually use this
structure when describing the linear maps between two vector spaces.
In particular we claim
$$
\mathcal{L} \left(V, W\right) = \left\{\text{linear maps }M: V \rightarrow W\right\} =
    W \otimes V^{*}
$$
Where $V^{*}$ indicates the dual space of $V$.

\subsection{Representing linear maps in some basis}
In a vector space once, we've set the basis vectors we can represent
all of the linear maps between these spaces. In particular, we can do
this via the Kronecker product
\begin{definition}[Kronecker Product]
\label{kronProd}
Given two vectors $v = \begin{pmatrix} v_1 \\ v_2 \\ \vdots \\ v_n \end{pmatrix}$, and
$w = \begin{pmatrix} w_1 \\ \vdots \\ w_m \end{pmatrix}$ we can define the tensor
product as
$$
    v \otimes w = \begin{pmatrix}
        v_1 w \\
        v_2 w \\
        \vdots \\
        v_n w
    \end{pmatrix} =
    \begin{pmatrix}
        v_1 w_1 \\
        v_1 w_2 \\
        \vdots \\
        v_n w_m
    \end{pmatrix}
$$
\end{definition}
We can also investigate how various operators combine within this notation. Say
that we were given four different operators paired across two different
vector spaces. We can establish that
$\left(A \otimes B\right) \left(v \otimes w\right) \equiv A v \otimes B w$.
Thus we end up with
$\left(C \otimes D\right) \left(A \otimes B\right) \equiv
CA \otimes DB$
We can also build up the Kronecker product for linear maps which has
a very similar form to how vectors combine.

An important piece of notation that we will use is
$\vec{\sigma}_1 \cdot \vec{\sigma}_2 \equiv
\sum_{i} \sigma_{i1} \otimes \sigma_{i2}$

\subsection{Decomposing tensor products}
In general, when taking tensory products between the states of two particles
you will not get an irreducible representation of your system. This is unless,
your particle is one dimensional. We would like to know how to decompose these
representaitons into an irreducible representaiton of the two particle systems.

Let's define $\rho_1 : G \rightarrow GL(n, \mathbb{C})$ and
$\rho_2 : G \rightarrow GL(m, \mathbb{C})$ as two representations
in two vector spaces. We can find the tensor product as
$\left(\rho_1 \otimes \rho_2\right)(g) \equiv \rho_1(g) \otimes \rho_2(g)$
and claim that this is a homomorphism. In other words, this is a representation
on $\mathbb{C}^n \otimes \mathbb{C}^m$. This mainly works
because $\rho_i$ are representations of $G$ in their repsective vector space.

\subsubsection{Example: Spin-$1/2$ decomposition}
Let's say we have the vector space $V = \left\{\Ket{\pm}\right\} \otimes 
\left\{\Ket{\pm}\right\}$. We will claim that $V = V_0 \oplus V_1$, where
$V_0 = \left\{\Ket{+-} - \Ket{-+}\right\}$ and 
$V_1$ is orthogonal to $V_0$, i.e.
$V_1 = \left\{\Ket{++}, \Ket{+-} + \Ket{-+}, \Ket{--}\right\}$.
We need to show that these bases are invariant under rotation in order to
show that they are irreducible representations. We can show this
using a "brute force" method where we apply the representaiton of rotation
onto the basis vectors.

Let's see how we can do this using antisymmetric tensors. Let's say that
$V_{ij} = -V_{ji}$. Then the action of $\rho$ will be
$V_{ij} \rightarrow \rho_k^i \rho_l^j V_{ij}$. Which there can only
be one linear invariant combination which is antisymmetric. We can then
analyze the symmetric portions of the space.

\section{Adding momenta}
Using this tensor product setup we can now analyze the addition
of the various momentums and how they add. In particular, we
want to represent the action of the entire momentum
using the tensor product on two different particles
$\rho \left( e^{- \frac{i}{\hbar} \vec{S} \cdot \hat{n} \theta}\right) =
\rho_1 \left(e^{- \frac{i}{\hbar} \vec{S}\cdot \hat{n} \theta}\right) 
\otimes \rho_2 \left(\dots\right)$. For small $\theta$ we will end up with
$\vec{J} = \vec{J}_1 \otimes 1 + 1 \otimes \vec{J}_2$. Similarly we can
find the form of $\vec{J}^2$ by accounting for the various
cross terms. Which ends up being
$\vec{J}^2 = \vec{J}_1^2 \otimes 1 +
1 \otimes \vec{J}_2^2 + 2 \left(\vec{J}_1 \otimes \vec{J}_2\right)$. We will
use a slight abuse of notation by dropping out the $1$ terms and implicitly know
that we are working in two different vector spaces.
We can manipulate the final term to be
$2 \left(\vec{J}_1 \otimes \vec{J}_2\right) =
2 J_{1z} \otimes J_{2z} + J_{1+} \otimes J_{2-} +
J_{1-} \otimes J_{2+}$

Using this we can then construct the various commutation relationships
in order to who the relationship between the various states. We end up
with the following
\begin{enumerate}
    \item $\left[J^2, \vec{J}\right] = 0$
    \item $J^2 \left(\Ket{+-} - \Ket{-+}\right) = 0$
    \item $J^2 \Ket{++} = 2 \hbar^2 \Ket{++}$
    \item $J_- \Ket{++} = \hbar \left(\Ket{-+} + \Ket{+-}\right) = 2\hbar^2 \dots$
\end{enumerate}

\section{Extending to infinite dimensions}
We now can extend the notion of tensor product to infinite dimensions
and to Hilbert spaces. This will give us a definition of entangeled states
\begin{definition}[Entangled State]
\label{entangledState}
If $\Ket{\psi} \in \mathcal{H}_A \otimes \mathcal{H}_B$, then we say
$\Ket{\psi}$ is entangled if $\Ket{\psi} \ne \Ket{\phi} \otimes \Ket{\eta}$,
for $\Ket{\phi} \in \mathcal{H}_A$ and $\Ket{\eta} \in \mathcal{H}_B$
\end{definition}
Effectively that the entangled states are irreducible states.

\begin{theorem}[Reduced Density Matrix Entanglement]
\label{redDensityMatrix}
Let $\rho_A = \tr \left(\Ket{\psi}\Bra{\psi}\right)$, and
$\rho_B = \tr_A \left(\Ket{\psi}\Bra{\psi}\right)$ be the reduced density matrices.
$\Ket{\psi}$ is entangled iff $S(\rho_A) = S(\rho_B) \ne 0$.
\end{theorem}

\subsection{Singular Value Decomposition}
Given $M \in \mathbb{C}^{n \times m}$, $MM^\dagger$ and $M^\dagger M$ are
Hermitian positive semi-definite matrices. These matrices can be diagonalized
to find their eigenvalues and vectors.
$MM^\dagger = V D V^\dagger$, where $VV^\dagger = I$ and $D$ is a diagonal
matrix with $N$ non-zero diagonal elements $\left\{\sigma_i\right\}_{i=1}^{N}$.
Similarly $M^\dagger M = U \tilde{D} U^\dagger$. Where $\tilde{D}$ will have the same
set of non-zero diagonal entries. The singular value decomposition of $M$ is
$M = U D V^\dagger = \sum_{i = 1}^{N} \sigma_i u_i v_i^t$. The number $N$ is called
the "rank" of the matrix.

We can then extend this definition to infinite dimensional spaces where we truncate
the sum up to the rank of the decomposition. This is also called the
{\color{red} Schmidtt decomposition}.

We can then represent a single state as
$\Ket{\psi} = \sum_{n,m} M_{nm} \Ket{\phi_n} \otimes \Ket{\eta_m} =
\sum_{k} \sigma_k \Ket{\tilde{\phi}_k} \otimes \Ket{\tilde{\eta}_k}$
If we have a normalizable state then $\Braket{\psi | \psi} = 1 \Rightarrow
\sum_k \sigma_k^2 = 1$. Which gives the partial trace equivalence as in 
\ref{redDensityMatrix}. In particular we get
$\rho_A = \tr_B \Ket{\psi}\Bra{\psi} = \sum_k \sigma_k^2 
\Ket{\tilde{\phi}_k}\Bra{\tilde{\phi}_k}$, and
$\rho_B = \tr_A \dots$

\end{document}
