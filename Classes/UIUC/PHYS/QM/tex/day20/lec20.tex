\section{Review of Lecture 19}
Last time we started talking about vector operators and how they will transform
under rotations. In particular we came up with the following definition
\begin{definition}[Vector Operator]
\label{vecOpt}
Let $R = R_{\hat{n}, \theta} \in SO(3)$. Then, a vector operator $\vec{V}$
transforms as
$$
\mathcal{D}^\dagger (R) \vec{V} \mathcal{D}(R) = R \vec{V}
$$
where $\mathcal{D} \approx 1 - \frac{i}{\hbar} \vec{J} \cdot \hat{n}\theta$ for
$\abs*{\theta} \ll 1$.
\end{definition}
This definition also implies the commutation relationship between any vector operator
and the generators of rotation
\begin{theorem}[Commutation relationships]
\label{vecOptCom}
If $\vec{V}$ is a vector operator then is has the following commutation relation
with the generators of rotation
$$
    \left[J_a, V_b\right] = i\hbar \epsilon_{abc} V_C
$$
\end{theorem}

\section{Cartesian Tensors}
We also decomposed the tensor products into direct sums of different vector spaces.
In particular we had the result $V_1 \otimes V_1 = V_2 \oplus V_1 \oplus V_0$.
We can use this decomposition to show how tensors within the product space
will break into pieces.

Let's say $T_{ij} = V_i U_j$ where $\vec{V}$ and $\vec{U}$ are vector operators.
This tensor $T$ is the cartesian product of the two vectors.
We can break this tensor into three different terms which have direct physical interpretations
$$
    T_{ij} = \vec{V} \cdot \vec{U}  + \epsilon_{ijk} \left(\vec{V} \times \vec{U}\right)_k
    \left(\frac{V_i U_j + V_j U_i}{2} - \frac{\vec{V}\cdot\vec{U}}{3} \delta_{ij}\right)
$$
The first term corresponds to the $V_0$ space, the second the $V_1$ space, and the third
coresponds to the $V_2$ space. Importantly the second term does not flip its sign
under parity transformations which makes it a type of pseudo-vector. Writing things
down in this form is a little bit cumersome so we would like a more understandable notation
which will give spherical tensors.

\section{Spherical Tensors}
We would like to write our tensor using the "master equation" which all spherical tensors
must satisfy.
\begin{align*}
\mathcal{D}^\dagger (R) Y_l^m \left(\vec{V}\right) \mathcal{D}(R) 
    &= Y_l^m \left(R \vec{V}\right) \\
    &= \sum_{m'} \mathcal{D}_{m'm}^{(l)} \left(R^{-1}\right) Y_l^{m'}\left(\vec{V}\right) \\
    &= \sum_{m'} \Bar{\mathcal{D}_{m'm}^{(l)} \left(R\right)} Y_l^{m'}\left(\vec{V}\right)
\end{align*}

\subsection{Example Spherical tensors for $l=1$}
We can typically write the spherical harmonics as follows which will give the following
spherical harmonics
\begin{align*}
    Y_{1}^0 = \sqrt{\frac{3}{4 \pi}} n_z &\rightarrow&
        T_0^{(1)} = \sqrt{\frac{3}{4 \pi}} V_z \\
    Y_1^{\pm} = \mp \sqrt{\frac{3}{4 \pi}} \frac{n_x \pm i n_y}{\sqrt{2}} &\rightarrow&
        T_{\pm 1}^{(1)} = \mp \sqrt{\frac{3}{4 \pi}} \frac{V_x \pm i V_y}{\sqrt{2}}
\end{align*}
A notational difference is setting $l=k$ and $m=q$. Previously we also looked at the Lie
Algebra in order to figure out the matrix elements of $\mathcal{D}(R)$ and found that
$\mathcal{D}_{qq'}^{(k)} =
\Bra{kq} e^{- \frac{i}{\hbar} \vec{J} \cdot \hat{n} \theta} \Ket{kq'}$.
By linearizing this definition and the LHS and RHS of the "master equation"
we can find the following commutation relationships
\begin{gather}
    \left[\vec{J} \cdot \hat{n}, T_q^{(k)}\right] = \sum_{q'}
        \Bra{kq'} \vec{J} \cdot \hat{n} \Ket{kq} T^{(k)}_{q'} \\
    \left[J_z, T_q^{(k)}\right] = q \hbar T_q^{(k)} \\
    \left[J_{\pm}, T_q^{(k)}\right] = \sum_{q'} \Bra{kq'} J_{\pm} \Ket{kq} T_{q'}^{(k)} =
        \hbar \sqrt{(k\mp q) (k \pm q + 1)} T_{q\pm1}^{(k)}
\end{gather}
This suggests an analogy between this spherical tensor $T$ and the spin states,
since the commutation relationship suggests that $T$ is an eigenstate of $J_z$
with eigenvalue $q \hbar$. In addition these operators have very similar properties
to the angular momentum of spin.

We can also mimic the calculation of Clebsh-Gordon coefficients using spherical
tensors. Specifically we know that the raising and lowering operators must have the following
$J_{\pm} - J_{1\pm} - J_{2\pm} = 0$. Using this we can calculate the corresponding
Clebsh-Gordon coefficients by sandwitching it between the appropriate state
$$
    \Bra{j',m'} J_{\pm} - J_{1\pm} - J_{2\pm} \Ket{j,k;m,q} = 0
$$
The spherical tensors act in a very similar way under the idea that
$T_q^{(k)} \Ket{j,m} \approx \Ket{j,m} \otimes \Ket{k,q}$
which suggests that the spherical tensors have a similar recursion relationship
to the Clebsh-Gordon coefficients. There's a very similar relationship
for $J^2$ as with the spin states. In particular
$$
    \sum_i \left[J_i, \left[J_i, T_q^{(k)}\right]\right] = \hbar^2 k \left(k+1\right) \Ket{kq}
$$
Which maps to $J^2 \Ket{kq} = \hbar^2 k \left(k+1\right) \Ket{kq}$ for spin states.
This action largely comes from Wigner-Eckart theorem.

\subsection{Wigner-Eckart Theorem}
\begin{theorem}[Wigner-Eckart Theorem]
\label{wigeck}
Let $T_q^{(k)}$ be a spherical tensor operator then, the matrix elements of the spherical
tensor between two states $\Ket{\alpha',j',m'}$ and $\Ket{\alpha,j,m}$ with additional
quantum numbers $\alpha$ and $\alpha'$ is
$$
    \Bra{\alpha',j',m'} T_q^{(k)} \Ket{\alpha,j,m} =
        \Braket{j,k;m,q | j,k;J=j',M=m'} \times
        \frac{\left<\alpha',j' \lVert T^{(k)} \rVert \alpha, j\right>}{\sqrt{2j+1}}
$$
The value of $\left<\alpha',j' \lVert T^{(k)} \rVert \alpha, j\right>$ is independent
of the value of $q$ or any particular basis. So we can evaluate it once
\end{theorem}
This gives us selection rules based on the Clebsh-Gordon coefficients.
\begin{lemma}
\label{wigeckSelection}
We get the following selection rules for Wigner-Eckart Theorem (\ref{wigeck}).
$\Bra{\alpha',j',m'} T_q^{(k)} \Ket{\alpha,j,m} = 0$ if 
$j' \notin \left\{j+k, j+k-1, \dots \abs*{j-k}\right\}$ or
$m' \neq m + q$. These come from the selection rules for the Clebsh-Gordon coefficients
\end{lemma}

\subsection{Example: Wigner Eckart Selection Rules}
\subsubsection{Example 1}
The first example will be finding
$\Bra{\alpha',j',m'} T_0^{(0)} \Ket{j,m} = 
    \Braket{j,0;m,0 | j',m'} \dots = \delta_{jj'} \delta_{mm'} \dots
$

\subsubsection{Example 2}
The second example is to take the spin of a particle which is restricted to
\newline
$\left\{\Ket{s,+s}, \Ket{s, s-1}, \dots, \Ket{s, -s}\right\}$, or the span of
$\left\{\Ket{+}, \Ket{-}\right\}$ over $\mathbb{C}$. We can construct
any spherical tensor by seeing how the operator changes the basis elements of our space.
In other words we can project $T_q^{(k)}$. In \ref{wigeck} the basis dependence is from the
Clebsh-Gordon coefficients. Which means that for fixed $k$ the various spherical tensors are
proportional to each other in this restricted space. This is why we found
$\vec{\mu} \propto \vec{S}$ in the beginning of the course.

\subsection{Projection Operators}
Within this restricted space from example 2 due to this proportionality we must have
some component of $\vec{V}$ along some generic spin operator $\vec{J}$. This
results in the projection theorem
\begin{theorem}[Projection Theorem]
\label{wikeckProj}
The matrix elements of $\vec{V}$ can be written as follows when within the restricted space
$$
    \Bra{\alpha',j,m'} V_q \Ket{\alpha,j,m} =
    \frac{\Bra{\alpha',j,m} \vec{J} \cdot \vec{V} \Ket{\alpha,j,m}}{\hbar^2 j(j+1)}
$$
\end{theorem}

\subsubsection{Homework \#10 Help}
For this week's problem set we need to write out some...
For this problem we will need the following theorem
\begin{theorem}[Addition Theorem]
\label{sphericalAddThm}
Let $T^{(k_1)}$ and $T^{(k_2)}$ be spherical tensor operators. Then,
$$
    T^{(k)}_{q} = \sum_{q_1=-k_1}^{k_1} \sum_{q_2=-k_2}^{k_2}
      \Braket{k_1,k_2; q_1 q_2 | kq} T_{q_1}^{(k_1)} T_{q_2}^{(k_2)}  
$$
where $k = k_1 + k_2$
\end{theorem}

As an example let $\vec{U}$ and $\vec{V}$ be vector operators. Then, we can find $T^{(2)}$ in
terms of the rank 1 spherical tensors given by $\vec{U}$ and $\vec{V}$.
See Sakari 3.469 (in the third edition) for explicit evaluations of this setup.

%\chapter{Chapter 4: Symmetry}
\section{Central Force Problem Extra Symmetry}
Previously we showed that in the hydrogen atom there was some kind of extra symmetry
that we couldn't account for with just the rotational symmetry. We'll dig a little bit
deeper into the source of that here. In classical mechanics recall that
$$
\dot{\vec{p}} = f(r) \hat{r} \text{ in spherical coordinates}
$$
For the correct selection of coordinates we can see that this is a total derivative
which greatly simplifies the problem. We can show this using the definition
of angular momentum
\begin{align*}
    \dot{\vec{p}} \times \vec{L} 
    &= f(r) \hat{r} \times \left(\vec{r} \times \vec{p}\right) \\
    &= m \frac{f(r)}{r} \vec{r} \times \left(\vec{r} \times \dot{\vec{r}}\right) \\
    &= \frac{m f(r)}{r} 
        \left[
            \vec{r} \left(\vec{r} \cdot \dot{\vec{r}}\right) - 
            \dot{\vec{r}} \left(\vec{r} \cdot \vec{r}\right)
        \right] \\
    &= \frac{m f(r)}{r} \left[
        \vec{r} \frac{1}{2} \diff{}{t} \left(\vec{r} \cdot \vec{r}\right) -
        \dot{\vec{r}} r^2
        \right] \\
    &= \frac{m f(r)}{r} \left[\vec{r} r \dot{r} - \dot{\vec{r}} r^2\right] \\
    &= m f(r) r^2 \left[
        \frac{\dot{r} \vec{r}}{r^2} - \frac{\dot{\vec{r}}}{r}
        \right] \\
    &= - m f(r) r^2 \diff{}{t} \left[\frac{\vec{r}}{r}\right]
\end{align*}
In the case that $f(r) = -k / r^2$, then $f(r) r^2 \propto \text{const.}$ and
$V(r) = -k / r$. In this case we get the following simplification
$$
    \diff{}{t} \left[ \vec{p} \times \vec{L} - m \frac{k \vec{r}}{r}\right] = 0
$$
This quantity $\vec{A}$ is the Laplace-Runge-Lenz vector which is conserved
for specific central force problems. With this conserved quantity
we have six degrees of freedom to work with. This is still in CM though and it would
be good to have a quantized version.

\section{Quantum Mechanics CFP}
The QM analog of the Laplace-Runge-Lenze vector is the following operator
$$
    \vec{M} = \frac{1}{2m} \left[
        \vec{p} \times \vec{L} - \vec{L} \times \vec{p} - \frac{Z e^2}{r} \vec{r}
    \right]
$$
We also have the following exercise problems and properties
\begin{itemize}
    \item $\vec{M} \cdot \vec{L} = 0 = \vec{L} \cdot \vec{M}$
    \item $M^2 = \frac{2}{m} H \left(L^2 + \hbar^2\right) + z^2 e^4$
    \item $\left[M_a, L_b\right] = i\hbar \epsilon_{abc} M_c$
    \item $\left[M_a, M_b\right] = -i \hbar \epsilon_{abc} \frac{2}{m} E_n L_c$
\end{itemize}
Up until the final commutation relationship $\vec{M}$ is a closed relationship. However,
since the final commutation relationship depends on the energy and thus, the Hamiltonian.
It's much easier to work with closed operators so we would like to somehow normalize this
operator $\vec{M}$. We can fix this by redefining $\vec{M}$ into the following
$$
    \vec{N} = \sqrt{\frac{m}{-2 E_n}} \vec{M}
$$
With the following commutation relationships
\begin{itemize}
    \item $\left[N_i, L_j\right] = i\hbar \epsilon_{ijk}$
    \item There is another one to write down
\end{itemize}
