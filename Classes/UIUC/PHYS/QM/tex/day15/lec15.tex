\section{Rotation matrices}
We have actual notes this time on canvas! Last time we realized that $\mathcal{H}_l$
can be represented by the spherical harmonic functions. In particular we can define
these representations via the map $\rho^{\left(l\right)}: SO(3) \rightarrow GL(2l+1, \mathbb{C})$.
The basis elements of this map are the three spherical harmonics
$Y_1^0$, and $Y_1^\pm$.

When rotating around $\hat{z}$, $\theta \mapsto \theta$ and $\phi \mapsto \phi + \eta$.
From this we know that only $Y_1^\pm$ is going to be modified by picking up a phase.
We'll then know the matrix $\rho^{(l)}$ based on how this modifies the function.

\section{Representations and their meaning}
Finding a representation of a Lie Group can be quite tedious as shown in the homework.
Instead it is much easier to work with a related concept to the representors of
$SU(2)$. This is what's typically done in QM work.

To start recall that $SU(2)$ is just the three sphere: $S^3 \cong SU(2)$. We also
know that $S^3 /\mathbb{Z}_2 \cong SO(3 \cong \mathbb{R} P^3$, where
$\mathbb{R} P^3$ is the real projective space. We can then work with the tangent
space fixed around the identity of $S^3$. We can then parameterize the manifold
around this point using a chart. Since the only difference between
$SU(2)$ and $SO(3)$ is modding out the negative identity, they will end up
having identical tangent spaces between the two. They will end up having isomorphic
\textbf{vector space structure} and \textbf{tangent space algebra}
(Lie algebra). We will denote the algebra using lowercase letters $su(2) \cong so(3)$.
As a result they will have the same kinds of generators, which are the angular momentum
operators.

Importantly you can take the exponential of the generators to map from the Lie algebra
to the "universal cover" of your Lie Group. As an example
$e^{-i \theta \vec{J}\cdot\vec{n}}$ is the representor of the $SU(2)$ group based
on the specific rotational operator we use. 

\section{Covers of $SU(2)$}
Let $\vec{J} = \frac{\hbar}{2} \vec{\sigma}$, with the following
commutator
$$
    \left[J_i, J_j\right] = i \hbar \epsilon_{ijk} J_k
$$
Whenever we work in higher dimensions we want spin to have these
same commutation relationships. Thus, we want to have a homomorphism
that preserves the group structure. In other words a map from the Lie
algebra to the Lie Group

Def: A Lie algebra is a vector spae $\mathcal{L}$ endowed with a Lie bracket 
$[\;,\;]: \mathcal{L} \times \mathcal{L} \rightarrow \mathcal{L}$, s.t.
the bracket is anti-commutive, linear in the first argument, and 
satisfies the Jacobi identity, ($\forall a, b, c \in \mathcal{L}$,
$\left[\left[a, b\right], c\right] + \text{cyc. permutations} = 0$.

A good example of this is the cross product. Thus, the cross product is nothing
but a Lie algebra on a vector space! Which is also why it doesn't generalize to higher
dimensions.

Def: Endomorphism, denoted 
$\text{End}(\mathbb{C}^n) = \left\{\text{all linear maps on } \mathbb{C}^n\right\}$
(without requiring an inverse exists) which has a commutator defined by the Lie bracket.

Def: A Lie algebra homomorphism is a $\phi: \left(\mathcal{L}_, \left[,\right]\right) \rightarrow
 \left(\mathcal{L}_2, \left[,\right]\right)$ is a linear map
that preserves the Lie bracket. I.E. $\forall a, b \in \mathcal{L}_1$,
$\phi \left(\left[a, b\right]_1\right) = \left[\phi \left(a\right), \phi(b)\right]_2$

We're working through all of these definitions to construct a mapping from
$su(2)$ to all Endomorphisms of $\mathbb{C}^n$. In other words,
the map $su(2) \rightarrow \text{End} \left(\mathbb{C}^n\right)$. We can map this
via the following exponential map,
$$
\rho_{SU(2)} \left(e^{- \frac{i}{\hbar} \vec{J} \cdot \vec{n}}\right) =
    \exp \left(- \frac{i}{\hbar} \theta \rho_{su(2)} \left(\vec{J}\right) \cdot \vec{n}\right)
$$
We've worked through the left hand side in HW 7 but it was quite tedious. Working with the
right hand side is much easier and will lead to the same result. Let's look at how this
works for spin!

\subsection{Exponential maps for Spin}
First off let's recall
$$
    J^2 \equiv J_1^2 + J_2^2 + J_3^2
$$
Since $J_i$ is Hermitian $J^2$ is also Hermitian. There are two important facts about this
operator
\begin{enumerate}
    \item $J^2$ is positive semi-definite
    \item $\left[J^2, \vec{J}\right] = 0$. Meaning that it commutes will all other operators.
        We call $J^2$ operators the Casimir operator
\end{enumerate}
From here we can arbitrarily choose one of the three $J_i$ to be simultaneously diagonalized
as $J^2$. This choice will set our convention for what spins we work with. We'll follow
typical convention and choose $J_z$. Let $\Ket{\lambda,\alpha}$ be s.t.
$J^2 \Ket{\lambda,\alpha} = \lambda \Ket{\lambda,\alpha}$ and 
$J_z \Ket{\lambda,\alpha} = \alpha \Ket{\lambda,\alpha}$. We want to find what
the values of $\lambda$ and $\alpha$ are just using these commutation relationships.

We can approach this by using ladder operators by defining
$J_\pm = J_x \pm i J_y$. You can show the following properties
using the known commutator
\begin{enumerate}
    \item $\left[J_+, J_-\right] = 2 \hbar J_z$
    \item $\left[J_z, J_\pm\right] = \pm \hbar J_\pm$
    \item $\left[J^2, J_\pm\right] = 0$
    \item $J_\pm J_\mp = J^2 - J_z^2 \pm \hbar J_z$
    \item $J^2 - J_z^2 = \frac{1}{2} \left(J_+ J_- + J_- J_+\right)$
\end{enumerate}
All of these can be shown via the commutation relationships. Since $J^2$ is positive
semi-definite we can put constraints on $\alpha$ and some higher state $\beta$ such that
they are annilated by the raising/lowering operator. This mainly comes from the last
property in which the left and right hand side must be positive semi-definite. 

We can find these constraints by looking at the state
$J_+ \Ket{\lambda,\alpha} = 0$ and $J_- \Ket{\lambda,\beta} = 0$. Then by taking
the expectation we can find two coupled equations
\begin{align}
    \lambda - \alpha^2 -\hbar \alpha = 0 \\
    \lambda - \beta^2 + \hbar \beta = 0
\end{align}
We will end up with two different conditions: $\alpha = -\beta$ and
$\alpha = \beta - \hbar \Rightarrow \alpha < \beta$. Since we are looking
for $\alpha$ to be the top value we will disregard the second condition.
From this we can find the quantization condition (based on the difference
between $\alpha$ and $\beta$) and find the values of $\lambda$ and $\alpha$.
In particular $\alpha = \hbar n /2$, and $\lambda = \hbar^2 j(j+1)$ where
$j = n/2$. We end up with a dimensionality of $2j+1 = n+1$ since these are all
of the possible states. 

To summarize exponentiating the Lie algebra may not give you the original
group you were working with but will give you a universal cover of that group
(a simply connected manifold of that space).

\subsection{Examples}
Show $J_\pm \Ket{j,m} = \sqrt{(j\mp m)(j\pm m + 1)} \hbar \Ket{j,m\pm1}$.

We'll start by exponentiating our representation
\begin{align*}
    \Bra{j',m'} e^{-\frac{i}{\hbar} \rho \left(\vec{J}\right) \cdot \hat{n} \phi}
        \Ket{j,m} = \delta_{jj'} 
        \Bra{j,m'} e^{-\frac{i}{\hbar} \vec{J} \cdot \hat{n} \phi} \Ket{j,m}
\end{align*}
We're going to omit the representation function $\rho$ because it's a bit
tedious to write down so it will be implicit in the exponential. We'll
also define this matrix element to be
$\mathcal{D}_{m'm}^{(j)} \left(R \hat{n}, \phi\right)$. Depending
on what you are looking at these elements can be really easy to calculate. Such as
rotating along $\hat{z}$ by some angle.
$$
    \mathcal{D}_{m'm}^{(j)} = e^{-im \phi} \delta_{j 0}
$$
Finding any further elements will require an understanding that we can decompose
any rotation via the Euler angles. Since two of these rotations are along $\hat{z}$
they will already be diagonalized. We can then further write down these rotation elements using
\begin{align*}
    \mathcal{D}_{m'm}^{(j)} \left(\alpha, \beta, \gamma\right) \equiv
        \Bra{j,m'} e^{-\frac{i}{\hbar} J_z \alpha}
        e^{-\frac{i}{\hbar} J_y \beta} 
        e^{-\frac{i}{\hbar} J_z \gamma} \Ket{j,m} =
    e^{-\frac{i}{\hbar} \left(\alpha m' + \gamma m\right)}
        \Bra{j,m'} e^{-\frac{i}{\hbar} J_z \beta} \Ket{j,m}
\end{align*}
Thus, if you know how to rotate around the $y$-axis you can rotate in any desired way
in a quantum mechanical sense. Sakurai defines this inner product as $d^{(j)}_{m'm}(\beta)$.

We can show what the spin half matrix looks like as follows
$$
    \mathcal{D}^{\frac{1}{2}}(\alpha, \beta, \gamma) =
        \begin{pmatrix}
            e^{-\frac{\alpha}{2}i} & 0 \\
            0 & e^{\frac{\alpha}{2}i}
        \end{pmatrix}
        \begin{pmatrix}
            \cos \frac{\beta}{2} & -\sin \frac{\beta}{2} \\
            \sin \frac{\beta}{2} & \cos \frac{\beta}{2}
        \end{pmatrix}
        \begin{pmatrix}
            e^{- \frac{\gamma}{2} i} & 0 \\
            0 & e^{\frac{\gamma}{2}i}
        \end{pmatrix}
$$
In our problem set we will be explicitly calculating $d^{(1)}_{m'm}$

\section{Orbital Angular momentum}
In classical mechanics orbital angular momentum is $r \times p$. We want to do the same
analog in quantum mechanics. We can do this by taking the spherical harmonics as our basis.
We will be looking for
$$
    \mathcal{D}^{(l)} (R) Y_l^m(\vec{x}) = Y_l^m \left(R^{-1} \vec{x}\right)
$$
We can see if this definition works by exponentiating our angular momentum operator
on both sides of this expression. Expanding things out goes as follows,
\begin{align*}
    e^{-\frac{i}{\hbar} \left(\vec{L}\cdot \hat{n}\right)\phi} Y_l^m (\vec{x}) =
\end{align*}
