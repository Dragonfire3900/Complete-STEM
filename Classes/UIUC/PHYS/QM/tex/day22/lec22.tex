\section{Review of last time}
We're going to wrap up the parity operations today. 
Last time we described two different states
\begin{align}
    \Ket{1} = \frac{\Ket{S} + \Ket{A}}{\sqrt{2}} &
    \Ket{2} = \frac{\Ket{S} - \Ket{A}}{\sqrt{2}}
\end{align}
With the Hamiltonian
$$
    H = 
    \begin{pmatrix}
        E_0 & -\Delta \\
        -\Delta & E_0
    \end{pmatrix}
$$
Under these the two states will oscillate together

\section{Parity as a whole}
Here we go!
\begin{theorem}[Parity Selection Rules]
\label{paritySelectionRules}
If $\epsilon$ and $\mathcal{O}$ are even and odd operators under parity; i.e.
\begin{align*}
    \pi \epsilon \pi = \epsilon \\
    \pi \mathcal{O} \pi = - \mathcal{O}
\end{align*}
then, $\Bra{\alpha} \epsilon \Ket{\beta} = 0$ if $\Ket{\alpha}$ and $\Ket{\beta}$
have opposite parity and $\Bra{\alpha} \mathcal{O} \Ket{\beta} = 0$ if $\Ket{\alpha}$ and
$\Ket{\beta}$ have the same parity.
\end{theorem}
This can be proved by showing that the inner product with these operators
must be equal to their negative by inserting $\pi \pi$ (which is equal to one
since $\pi$ is an involution). 

This is the first selection rule which can give you the elements of various operators
without needing to directly computing anything at all! 
{\color{red} This can potentially save you a ton of work and you will be asked to use
these selection rules on HW/exams!}

\subsection{Example: Infinite Potential Array}
Let's say that we have an array of delta potentials at $x = (2n+1)/2$ with $n \in \mathbb{Z}$.
This can be treated as a sequence of particles in a box and we will call the state
which contains the origin $\Ket{0}$ and any other state with the $n$-th box the
$\Ket{n}$ state. Since each of the boxes are equivalent they have the
same energy, $H \Ket{n} = E_0 \Ket{n}$. However, this setup has a 
translational symmetry which means that $\Ket{n}$ is not an eigenstate of
$T(\alpha)$. We can construct an eigenstate though using the definition
$$
    \Ket{\theta} = \sum_{n=-\infty}^{\infty} e^{i \theta n} \Ket{n}
$$
We can show that under the translation operator we get the following
$T(\alpha) \Ket{\theta} = e^{-i \theta} \Ket{\theta}$. Which means that we can construct
simultaneous diagonalizations of $H$ and $T(\alpha)$. Under this diagonalization we can
connect the different states using this diagonalization.

Under the infinite well situation the two states do not really mix but we can relax this
assumption. Under a finite magnitude well we will have some mixing between the bins.
How far this mixing occurs depends on the assumptions of the crystal we make but a common
one is the Tight-binding approximation
\begin{definition}[Tight-binding approximation]
\label{tightbinding}
The tight binding approximation assumes that only neighboring cells are connected to one another
with some split potential. In particular the Hamiltonian is
$$
    H \Ket{n} = E_0 \Ket{n} - \Delta \Ket{n-1} - \Delta \Ket{n+1}
$$
\end{definition}

Under the tight-binding approximation we can calulate the energy of the
state $\Ket{\theta}$. In particular we can find that
$H \Ket{\theta} = (E_0 - 2 \Delta \cos(\theta)) \Ket{\theta}$. Using this setup
we can model the motion of electrons in a lattice. However, we need a model of something
like the plane wave solution for this setup. A physicist called "Bloch" modeled this
situation a while ago and demanded the following requirements for the
position representation $\Braket{x | \theta} = e^{ik x} U_k (x)$.
First it needed to be periodic $U_k(x+a) = U_k(x)$ where
$k = \theta / a$. Under this demand we can find the dispersion relationship
for this "Bloch wave" being
$$
    E(k) = E_0 - 2 \Delta \cos \left(k a\right)
$$
This setup generalizes the parity operator which we've discussed before.

\section{Time Reversal Symmetry}
Notation for this type of symmetry is pretty bad if you stick to
bra-ket notation. {\color{red}So we're going to use inner product notation instead
of bra-ket notation.}

\subsection{TR in Classical Mechanics w/ Conservative Force}
Under this setup we can write the EOM to be
$$
    m \ddot{\vec{x}} = - \vec{\nabla} V
$$
If the potential does not depend on $t$, in this case the setup is invariant under
time reversal. This is generally true for most systems. This is except for
heat diffusion which is not time reversible. Over time the heat will expand outwards
when going forward in time and shrink when going backward. This is largely
because heat diffusion only has a single derivative with respect to time
$$
    \diffp{\psi}{t} = \nabla^2 \psi
$$
We've made an analogy to Heat diffusion from the Schrodinger equation using a Wick
rotation. So we're going to analyze if this analogy fully works

\subsection{Quantum Mechanics TR}
Let's look at Schrodinger's equation for spinless particles. This ends up being
$$
    i \hbar \diffp{\psi}{t} = H \psi \left(\vec{x}, t\right)
$$
If we apply the operation $t \rightarrow -t$ we get a minus sign
on the LHS. However, there is a solution. Schrodinger's equation has a complex component
in front of the time differential. So if we take the complex conjugate we can find a
corresponding solution which follows the same Schrodinger's equation. In particular,
we can define a solution of $\psi_r (\vec{x}, t) \equiv \bar{\psi} (\vec{x}, -t)$
which follows the same Schrodinger equation under time reversal. Effectively the 
time reversal not only reverses time but also takes the complex conjugate.
\begin{definition}[Anti-Unitary Operator]
\label{antiUnitary}
An operator $A$ is called anti-unitary if $\forall \psi, \phi \in \mathcal{H}$
we have the following properties
\begin{itemize}
    \item $\left(A \phi, A \psi\right) = \left(\psi, \phi\right) = \bar{\left(\phi, \psi\right)}$
        Which means that they preserve the complex conjugate of the inner product.
    \item $A \left(c_1 \phi + c_2 \psi\right) = \bar{c}_1 A \phi + \bar{c}_2 A \psi$,
        $\forall c_1, c_2 \in \mathbb{C}$
\end{itemize}
\end{definition}
Let's look at an example of an anti-unitary operator. Let $K$ be the complex
conjugation operator in a fixed basis. The action of $K$ on the basis $\alpha \psi$
will be $K \left(\alpha \psi\right) = \bar{\alpha} \psi$
where $\alpha \in \mathbb{C}$. In other words the
complex conjugation will only affect the coefficients of the basis elements. 

\begin{theorem}[Anti-Unitary Product]
\label{antiUnitaryProduct}
If $A$ and $B$ are anti-unitary, then $AB$ is unitary
\end{theorem}
We can do this by looking at $\left(AB \phi, AB \psi\right)$ and simplifying it
using the properties of anti-unitary operators. In particular you can show that
this is equal to $\left(\phi, \psi\right)$ which means that $AB$ is unitary.
\begin{corollary}
Any anti-unitary operator $A$ can be written as, $A = UK$, where
$U$ is unitary and $K$ is the complex conjugation operator.
\end{corollary}
From \ref{antiUnitaryProduct} we know that $AK = U$ which we can act on by the inverse of
$K$. Since $K$ is anti-unitary it's inverse it itself which implies that
$A = UK$.

As we mentioned earlier in order to get time-reversal we need to take the
complex conjugate and map $t \rightarrow -t$. Motivated by this we can
define the time reversal operator.
Let $\Theta$ be an anti-unitary operator (time reversal operator) s.t.
$$
    \Ket{\psi_r, t} = \Theta \Ket{\psi, -t} = 
    \Theta e^{\frac{i}{\hbar} H t} \Ket{\psi, 0}
$$
Contrary to the name what this operator does is flip the spin states and doesn't explicitly
reverse the time. In particular, it is a solution to the Schrodinger's equation evaluated
at $-t$. Since this state satisfies the Schrodinger equation it must evolve in the exact same
way. In other words $\psi_r$ evolves as $\Ket{\psi_r, t} = e^{-\frac{i}{\hbar}Ht} \Ket{\psi_r, 0}$
and thus,
$$
e^{-\frac{i}{\hbar} H t} \Ket{\psi_r, 0} = e^{- \frac{i}{\hbar} H t} \Theta \Ket{\psi, 0}
$$
Which leads to
$$
    \Theta e^{\frac{i}{\hbar} H t} = e^{- \frac{i}{\hbar} H t} \Theta
$$
for $t \ll 1$, $\Theta i H = - i H \Theta$, which means that for a linear
time reversal operator $\left\{\Theta, H\right\} = 0$ which means that
this operator anti-commutes with the Hamiltonian. However, this ends up being
a problem because this setup implies that the energy is bounded from above and
below in the same way (if one is bound so is the other and if one isn't then same with the other).
As a result we cannot conclude that $\Theta$ must be an anti-linear operator.

If $\Theta$ is anti-linear then, $\left[\Theta, H\right] = 0$. In addition, since we
want this operator to preserve probability we can conclude that $\Theta$ must be
an anti-unitary operator.

Since $\Theta$ commutes with $H$ we can simultaneously diagonalize them. Thus, the state
$\Theta \Ket{\psi}$ must have the same energy eigenvalue. Thus, if $\Ket{\psi}$ is a
non-degenerate energy eigenstate, then $\Theta \Ket{\psi} = e^{i \delta} \Ket{\psi}$.
This has some interesting consequences in justifying why the SHO states are always real,
\begin{align*}
    e^{-i\frac{\delta}{2}} \Theta \Ket{\psi} &= e^{i \frac{\delta}{2}} \Ket{\psi} \\
    \Theta e^{i \frac{\delta}{2}} \Ket{\psi} &= e^{i \frac{\delta}{2}} \Ket{\psi} \\
    \Theta \Ket{\phi} &= \Ket{\phi}
\end{align*}
Thus we must have a real state $\Ket{\phi}$.

\subsection{Inner Product Implications}
For any linear operator $A$, consider the inner product $\left(\psi, A \phi\right)$.
The matrix element will typically be written as $\Bra{\psi} A \Ket{\phi}$ but
with anti-unitary operators it matters which state $A$ is operating on. Since
$\Theta$ is an anti-unitary operator this inner product is equal to
$\left(\Theta A \phi, \Theta \psi\right)$
\begin{align*}
    \left(\psi, A \phi\right)
    &=
        \left(\Theta A \phi, \Theta \psi\right) \\
    &= \left(\Theta A \Theta^{-1} \Theta \phi, \Theta \psi\right) \\
    &= \left(\Theta \phi, \Theta A^\dagger \Theta^{-1} \Theta \psi\right)
        \text{ the operator } \Theta A \Theta^{-1} \text{ is linear} \\
    &= \left(\tilde{\phi}, \Theta A^\dagger \Theta \tilde{\psi}\right)
\end{align*}
From classical physics we want $\left(\phi, \vec{x} \phi\right) = 
\left(\tilde{\phi}, \vec{x} \tilde{\phi}\right)$. From the previous
derivation we get $\left(\tilde{\phi}, \Theta \vec{x} \Theta^{-1} \tilde{\phi}\right)$
As a whole this implies that,
$\Theta \vec{x} \Theta^{-1} = \vec{x}$ which means the $\left[\Theta, \vec{x}\right] = 0$.
By the same computation for momentum we must demand that
$\left(\phi, \vec{p} \phi\right) = - \left(\tilde{\phi}, \vec{p} \tilde{\phi}\right)$
which means that $\left\{\Theta, \vec{p}\right\} = 0$

If we want these representations to commute we want $\left[T, R\right] =0$,
where $T$ is the time reversal operator. This implies that
$\left[\Theta, e^{- \frac{i}{\hbar} \vec{J} \cdot \hat{n} \theta}\right] = 0$
We can then use this property to show that the time reversal operator
we have $\left\{\Theta, \vec{J}\right\} = 0$ using the small angle approximation.

\subsection{Specific Dimensions}
We can now look at how the time reversal operator works in specific dimensions with
some amounts of spin. In particular we can show that in 1-dim with spin-0 we get
$\Theta \Ket{\psi} = e^{i \delta} \Ket{\psi}$ which implies that $\Theta^2 = 1$.
which is a nice representation. However, this nice properties will not always be true.

However, for spin-$1/2$ states we will end up needing to use a double cover of 
$SO(1, 3)$. Let's work through it! Let's look at dim-2 for spin-$1/2$.
We can start by acting on $S_z \Theta \Ket{+} = - \Theta S_z \Ket{+} = 
- \hbar/2 \Theta \Ket{+} = -\hbar /2 \eta \Ket{-}$. In general, we can
find $\Ket{\hat{n}, +} = e^{-\frac{i}{\hbar} S_z \alpha} e^{- \frac{i}{\hbar} S_y \beta} \Ket{+}$
and then act on it by $\Theta$.
\begin{align*}
    \Theta \Ket{\hat{n}, +} = e^{- \frac{i}{\hbar} S_z \alpha} e^{- \frac{i}{\hbar} S_y \beta}
        \Theta \Ket{+} = e^{- \frac{i}{\hbar} S_z \alpha} e^{- \frac{i}{\hbar} S_y \beta}
        \eta \Ket{-} = \eta \Ket{\hat{n}, -}
\end{align*}
At the end we get the following representation for $\Theta$
$$
    \Theta = \eta e^{- \frac{i}{\hbar} \pi} K
$$
We will finish this calculation on Wednesday

