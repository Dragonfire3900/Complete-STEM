\section{Boosts cont.}
Def: $\Lambda \in SO(1,3)^+$ is a pure boost if for some $\vec{n}\in \mathbb{R}^3$
(spatial) s.t. $\Lambda$ leaves the plant $\perp$ to $\vec{n}$ invariant.

Last time we defined what it means to boost purely along the $\hat{x}$ axis
in terms of the rapidity $\chi$ defined by $\frac{v}{c} = \tanh \chi$. This
was defined using the matrix in the previous section.

There is a very fast way to diagonalize this matrix by recognizing that
$\Lambda_{\hat{x}} = \cosh \chi I_{2\times 2} + \sinh \chi \sigma_x$.
We know how to diagonalize the Pauli matrix $\sigma_x$ due to the previous
homework.

We won't be using boosts too much but we do need it to understand the connection
between $SO(1, 3)$ and $SL(2)$ as a double cover.

Lemma:
\begin{enumerate}
    \item $\Lambda \in O(1, 3) \Rightarrow \Lambda^{-1} \in O(1,3)$. This is also
        true in $SO(1,3)$ and $SO(1,3)^+$
    \item $\Lambda \in O(1, 3) \Rightarrow \Lambda^t \in O(1,3)$. Similarly for
        $SO(1,3)$ and $SO(1,3)^+$
\end{enumerate}
Thus, taking the inverse and transpose are contained within the groups and are closed.
The first statement can be proven by realizing $\Lambda^t g \Lambda = g$ by definition
and that the inverse exists. In addition, it's also important to realize that taking
the inverse is a continuous map. Thus, the inverse can only map to connected components
in the space. 

Due to this the elements of the Lie groups stay in the same group. We're showing this
because we want to prove the following

Th: $\Lambda \in SO(1, 3)^+ = (\text{Boost})(\text{Rotation}) \equiv \sqrt{\Lambda \Lambda^t} U$
Here's a sketch of the proof. Recall that complex numbers have a polar decomposition
$z = \abs*{z} e^{i \phi}$. We can define the polar decomposition using the final part of Th.
Importantly since $\Lambda \in SO(1,3)^+$ the product in the square root is positive
definite which means that the square root is well defined. We now need to find what
$U$ is. By definition $U \equiv \left(\Lambda \Lambda^t\right)^{-1/2}$. We can
take the product $U^t U = 1$ to show that $U$ is by definition a pure rotation.
The remaining part is to show that $\sqrt{\Lambda \Lambda^t}$ is a boost by
the definition. To do this we need another Lemma.

Lemma: If $\Lambda \in SO(1, 3)^+$ is symmetric, then if
$\lambda$ is an eigen value of $\Lambda$ with eigen vector $v$, then
$\frac{1}{\lambda}$ is an eiven value of $\Lambda$ with an eigen vector $gv$.

Pf.
\begin{align*}
    \Lambda g \Lambda = g \\
    \Lambda v = \lambda v \text{ But. } \\
    gv = \Lambda g \Lambda v = \lambda \Lambda g v \\
    \Downarrow \\
    \Lambda \left(gv\right) = \frac{1}{\lambda} \left(gv\right)
\end{align*}

Another needed Lemma is
Lemma: If $\Lambda \in SO(1,3)^+$ is symmetric and positive definite, then
$\Lambda$ is a boost.

Pf.
Let $\lambda_n$ be the eigen values of $\Lambda$ with e.vecs $v_n$,
pos. def $\Rightarrow \prod_n \lambda_n = 1$, $\lambda_n > 0$. An important
idea to realize is that $\lambda = 1$ implies that $v$ is invariant under the action
of this operator. Which would be an invariant subspace.
\begin{align*}
    v_i^t \Lambda g \Lambda v_i = v_i^t g v_i \\
    \dots = v_{i_{\mu}}^2 \\
    \lambda_i^2 v_{i_{\mu}}^2 = \dots \\
    \therefore \; \left(\lambda_i^2 - 1\right) v_{i_{\mu}}^2 = 0
    \\\Downarrow\\
    \text{either } \lambda_i = 1 \text{ or } v_{i_{\mu}}^2 = 0
\end{align*}
If $\lambda_i = 1 \; \forall i$, then $\Lambda = I$.
Otherwise, assume $\lambda_1 = \frac{1}{\lambda_2} \pm 1$
Let $v_1 = \frac{1}{\sqrt{2}} \begin{pmatrix} 1 \\ \vec{W} \end{pmatrix}$, 
$\lVert \vec{W} \rVert_2 = 1$. This implies that
$v_2 = gv_1$. You can also show that the other two eigenvectors
are perpendicular to $v_1$ and $v_2$ by imposing that $v_1^t v_2 = 0$.
Since $\Lambda \Lambda^t$ is positive definite and symmetric is satisfies this Lemma.

\section{Double Cover of $SL(2)$}
It's this polar decomposition which allows us to show the double cover between $SO(1,3)$ and
$SL(2)$. First let's start with a definition.
$$
    H_2 = \left\{ \text{ all } 2\times2 \text{ Hermitian matrices over } \mathbb{C}\right\}
$$
Using this definition and the properties of a Hermitian matrix we can conclude
that the diagonals are always real and that the anti-diagonals are complex conjugates
of each other. Using this we can then parameterize this matrix using the isomorphism
$$
    x \rightarrow \hat{x} = 
    \begin{pmatrix} x_0 + x_3 & x_1 - i x_2 \\ x_1 + i x_2 & x_0 + x_3 \end{pmatrix} =
    x_{\mu} \cdot \sigma_{\mu}
$$
This is a map between $\mathbb{R}^{4n}$ and $H_2$ but we would like to have a reverse
map. We're doing this to map Lorentz transformations to the Hilbert matrix space. This
way we can boost operators in Hilbert space for $\mathbb{R}^{4n}$

Def: Frobenius inner product on $H_2$ is $\forall A, b \in H_2$,
$\left<A, B\right>_{F} = \frac{1}{2} \text{Tr}\left(A^\dagger B\right)$
{\color{red}(in infinite dim. this is called Hilbert-Schmidt inner product)}

This inner product for the $\sigma_i$ matrices have a nice property with the
Frobenius inner product. In particular,
$\left<\sigma_{\mu}, \sigma_{\nu}\right> = \delta_{\mu \nu}$. It's also important to note
that the identity matrix has been used as $\sigma_0$.
Using this property we can actually define the reverse map between 
$H_2 \rightarrow \mathbb{R}^{4}$.
Using the following 
$$
    \hat{x} \rightarrow \left(\left<\sigma_i, \hat{x}\right>_{F}\right)
$$
It's this transformation on $H_2$ which allows us to relate things to Fermions
in $H_2$.

We originally defined the Lorentz transformations in terms of preserving the Minkowski
norm. We need an equivalent in $H_2$ in order to make sure that things make sense. There
are two things we want to preserve: $\det \hat{x} = x_{\mu}^2$, and the composition
between Lorentz boosts (which is a homomorphism). Effectively we want the following

Let $x \in \mathbb{R}^4$ and $x \rightarrow x' = \Lambda x$, where $\Lambda$ is a 
Lorentz boost, then we want to find, 
$A \in SL(2, \mathbb{C} = 
\left\{ 2 \times 2 \text{ matrices over } \mathbb{C},\; \det = 1 \right\}$, s.t.
$\hat{x'} = A \hat{x} A^\dagger$.

We can show that the determinant is preserved using $\det = 1$. We also
need to show that this is a homomorphism and what is needed to
do this is the adjoint in $\hat{x'}$.

Finally, we lso need to show that there is a unique $A$ for each Lorentz boost $\Lambda$.
This is where we will be going next. However, we can observe that $A$ and $-A$ both
are in $SL(2, \mathbb{C})$ and both have $\det = 1$. Thus, they are equivalent in terms
of their action on $\hat{x}$. As a result, we have a 2-to-1 correspondance between
$SL(2, \mathbb{C})$ and $SO(1, 3)^+$. The two covers we have of $SL(2)$ generate
Fermions as one of the covers will produce a sign flip under it's action. This corresponds
to the parity of Fermions.

\subsection{Double cover Proof}
Now we'll look at how this is generated by proving the following theorem
Th: $SL(2, \mathbb{C}) / \mathbb{Z}_2 \cong SO(1, 3)^+$

Lemma: 
$e^{i \vec{n}\cdot\vec{\sigma}} = 
\cos \theta + i \left(\vec{n}\cdot\vec{\sigma}\right) \sin \theta$, 
where $\lVert\vec{n}\rVert_2 = 1$,
$\vec{n} \in \mathbb{R}^3$. This comes from the involutive properties.

In general what we are getting $S^3 = SU(2)$ and 
$S^3 / \mathbb{Z}_2 = R \mathbb{P}^3 \cong SO(3)$.

Lemma:
For any $U \in U(n)$, $U = e^{i H}$, where $H = H^\dagger$.
If $U \in SU(n) \Rightarrow U = e^{i H}$, $H = H^\dagger$, and $U$ is traceless.
For any $H \in HPD(n)$ ($HPD$ = Hermitian positive definite), then
$H = e^K$, $K = K^\dagger$. With a similar relationship that the $SU(n)$ has.

pf.
For $A \in SL(2, \mathbb{C})$, then we can write $A = H U$, where
$H \in SHPD$, and $U \in SU(2)$. We can establish
$A = \sqrt{A A^\dagger} U$ which from here it follows the same proof
of the Lorentz polar decomposition proof. Using this $H$ corresponds to the
boost operation while $U$ corresponds to the rotation.

We need to show this though through the definitions of a rotation and a boost.
We need a Lemma for this

Lemma: Let $A \in SU(2, \mathbb{C})$, and 
$A = e^{-\frac{i}{2} \theta \vec{n} \cdot \vec{\sigma}} =
\cos \frac{\theta}{2} I - i \left(\vec{n}\cdot\vec{\sigma}\right) \sin \frac{\theta}{2}$.
$A \hat{x} A^\dagger$ rotates $x$ around $\vec{n}$ by $\theta$.

pf.
The three things we need to show are the invariance of the time component, any direction
along $\vec{n}$ is invariant, and the $\vec{b} \in \mathbb{R}^3$ ($\vec{b} \perp \vec{n}$)
rotate around by the correct angle. In particular,
$A \left(\vec{b} \cdot \vec{\sigma}\right) A^\dagger =
\cos \theta \left(\vec{b} \cdot \vec{\sigma}\right) +
\sin \theta \left(\vec{n}\times\vec{b}\right)\cdot \vec{\sigma}$

Lemma: $H = e^{\frac{\gamma}{2} \vec{n} \cdot \vec{\sigma}}$

pf.
Once again there are three different things you should be showing
\begin{enumerate}
    \item $\vec{b} \perp \vec{n} 
        \Rightarrow H \left(\vec{b}\cdot\vec{\sigma}\right) = \vec{b}\cdot\vec{\sigma}$
    \item $H \left(\vec{n}\cdot\vec{\sigma}\right) H^\dagger =
        \cosh \chi \left(\vec{n}\cdot\vec{\sigma}\right) + \sinh \chi x_0$
    \item $H H^t = \sinh \chi \left(\vec{n}\cdot\vec{\sigma}\right) +
        \cosh \chi \sigma_0$
\end{enumerate}
