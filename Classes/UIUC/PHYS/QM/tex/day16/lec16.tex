\section{Lie Aglebra $su(2)$ of $SU(2)$}
$\vec{S} = \frac{\hbar}{2} \vec{\sigma}$ satisfies 
$\left[S_j, S_k\right] = i \epsilon_{ikl} S_l$. In addition,
any $A \in SU(2)$ can be written as
$e^{-\frac{i}{\hbar}\phi \vec{S} \cdot \vec{n}}$,
$\hat{n}\in \mathbb{R}^3$, and $\abs*{\hat{n}} = 1$.
This is just a summary of last weeks work.

You can parameterize the Lie group elements by exponentiating
the Lie algebra. We have done this with pauli matrices as
they furnish a two dimensional cover. This is typically done
using a "highest weight representation" which orders the basis
from highest spin to lowest spin.

Last week we also computed the endomorphism from $su(2)$ to the complex
space. In other words $\rho: su(2) \rightarrow \text{End}(\mathbb{C}^{2l+1})$.
This gave rise to the spherical harmonics. In this class we will write this
down as $\rho(\vec{s}) = \vec{J}$. Where the dimensionality is implicit
in this notation.

\subsection{How does a quantum state transform?}
The main claim is that
$$
    \Ket{\psi'} = e^{-\frac{i}{\hbar} \vec{J}\cdot \hat{n} \theta}
    \Ket{\psi}
$$
This will rotate the quantum state around the axis $\hat{n}$ by an
angle $\theta$. There is a minor ambiguity in why there is a minus
sign in the exponential. Suppose we wanted to measure the spin along
an arbitrary direction
\begin{align*}
    \Bra{\psi'} \vec{J}\cdot \hat{a} \Ket{\psi'} \\
    \Bra{\psi} e^{\frac{i}{\hbar} \vec{J} \cdot \hat{n} \theta}
        \vec{J} \cdot \hat{a}
        e^{-\frac{i}{\hbar} \vec{J} \cdot \hat{n} \theta} 
        \Ket{\psi}
\end{align*}
Previously when working with rotations we had switched signs on the exponentials
We we can end up simplifying this to be the inverse of $R$.
\begin{align*}
    \Bra{\psi} \vec{J} \cdot \left(R^t_{\hat{n}, \theta} \hat{a}\right) \Ket{\psi} \\
    \Bra{\psi} \left(R_{\hat{n}, \theta} \vec{J}\right) \cdot \hat{a} \Ket{\psi}
\end{align*}
By taking the inner product with $\hat{a}$ of the difference between $\vec{J}$ and
$R \vec{J}$ we can find this to be zero.
Thus, we conlude that $\vec{J}$ must transform like a vector. We will extend this in
a few weeks to find "spherical tensors" which transform like higher order versions of
$\vec{J}$. For the upcoming homework we will be working on proving the addition theorem
which will be quite helpful in 581. Also somehow we're behind schedule?

\subsection{Question from last time}
The original $\delta_{mm'}$ was correct and the reason is that we
are exponentiating a diagonal matrix. He also asked us to correct him and ask
questions about the content as soon as possible

\section{Density Matrix}
We needed the Lie Algebra formalism in order to understand
this setup. We will also need the density matrix in order to understand
QIS. 

We're going to go back to the Stern-Gerlach experiment. As described before
we will get a stream of silver atoms and put them through an inhomogenous magnetic field.
We will observe two eigenstates of the $S_z$ operator. The question is:
what is the state of the experiment before the magnetic field?
In theory, we would expect them to not be polarized in any direction and should have
an equal chance of being in any direction. 

In quantum mechanics we never really measure a single particle and instead measure
a set of multiple particles or an "ensemble". 

\begin{definition}[Pure State]
\label{purestate}
A pure state $\Ket{\psi} \in \mathcal{H}$ is a single state within the Hilbert space
\end{definition}

However, we need to move away from this and towards describing how multiple particles
are within the entire hilbert space. For example, let's say $\Ket{\psi} = \Ket{\hat{x}, +}$
and compare it against a "mixed" ensemble of 
$$
    \left\{\left(\frac{1}{2}, \Ket{+}\right), \left(\frac{1}{2}, \Ket{-}\right)\right\}
$$
In other words there are two states which make up half the whole. One key difference
is that there will be interference in the $\Ket{\psi}$ state but not in the
ensemble state. If we measure these two systems along the $S_z$ they will appear
to be identical. However, if we measure along $S_x$ we will find only a single
beam in the $\Ket{\psi}$ state but two in the ensemble state. 

This second version describes a "mixed state" which is a set of individual states
which have a probability between them. From this we need to re-write our formalism
to account for these ensembles (how to take expectations, and represent individual vectors).

\subsection{Motivation behind the density operator}
The setup for this is the same as the typical S-G experiment. The difference is that
we will take the magnetic field along $\hat{n}$ and we will randomize this direction
for every particle. Then we will only keep the $\hbar/2$ states from this random experiment.
This will be functionally the same as the previous experiment. But in doing this we can 
describe how to "average" a different Hermitian operator $A$.

We will describe this using $\left[A\right]$ to denote the ensemble average of the
operator $A$. It will have the following definition
$$
    \left[A\right] = \frac{1}{4 \pi} \int d \Omega
       \Bra{\hat{n}\cdot\vec{S}, +} A \Ket{\hat{n}\cdot\vec{S}, +}
$$
After inserting the identity operator 
$\mathbb{1} = \sum_{n=0}^{\infty} \Ket{\psi_n}\Bra{\psi_n}$
Using this we can then describe this integral by treating this 
average as the density matrix $\rho$ of the projection operator.
\begin{align*}
    \left[A\right] = \sum_{n=0}^\infty \Bra{\psi_n}
        \left(\int d \Omega \frac{\text{state}}{4 \pi}\right) A \Ket{\psi_n} \\
    \therefore \; \left[A\right] = \text{tr}(\rho A)
\end{align*}
This has the same intuition as the previous setup with this new operator $\rho$.

\subsection{Mathematical Formalism}
\begin{definition}[Density Operator]
\label{denseop}
A density operator $\rho: \mathcal{H} \rightarrow \mathcal{H}$ that is Hermitian,
positive semi-definite, and $\tr(\rho) = 1$
\end{definition}
In general with this formalism we can write the density operator as the following
$\rho = \sum \omega_i \Ket{\phi_i}\Bra{\phi_i}$. 
Where $1 \ge \omega_i > 0$. This will map between
different states even in other bases. In a pure state we will have some of the
$\omega_i$ being zero which will make this operator semi-definite.

From these definitions we can establish the following properties of the
density operator
\begin{enumerate}
    \item $\left[A\right] \equiv \tr(\rho A) = \sum_i \omega_i \Bra{\phi_i} A \Ket{\phi_i}$
    \item $\rho$ is positive semi-definite
    \item Each eigen value, $\lambda$, of $\rho$ is bounded by $0 \le \lambda \le 1$.
        This comes from the trace being one and the positive semi-definite boundary.
    \item $\tr(\rho^2) \le 1$.
        This comes from the previous property
\end{enumerate}

\begin{definition}[Pure State]
\label{purestate2}
$\rho$ is called pure if, $\exists \Ket{\psi} \in \mathcal{H}$, s.t.
$\rho = \Ket{\psi}\Bra{\psi}$
\end{definition}

From this we have the following theorem
\begin{theorem}[Pure state properties]
\label{pureprops}
\begin{enumerate}
    \item $\rho^2 = \rho$ iff $\rho$ is pure
    \item $\tr(\rho^2) = 1$ iff $\rho$ is pure
\end{enumerate}
\end{theorem}

Let's consider the following example
Condier $\Ket{\psi_\pm} = \Ket{S_x,\pm}$. In this case $\rho = \Ket{\psi_\pm}\Bra{\psi_\pm}$.
We can find the matrix form by using vector notation
$$
    \rho = \frac{1}{2} \left(I \pm \sigma_x\right)
$$
We expect this to be a pure state. We can do this by doing one of the following:
diagonalize the matrix and show a non-trivial kernel, diagonalize the matrix and show that
it is an outer product, and use theorem \ref{pureprops}.

Let's look at the completely generic form of the density matrix for the two $\hat{z}$ spin
states. Let $\mathcal{H} = \text{span}_{\sigma} \left\{\Ket{+}, \Ket{-}\right\}$. From this
we know that $\rho = a I + \vec{b} \cdot \vec{\sigma}$. Combined with the trace
$a = \frac{1}{2}$. We now only need to diagonalize the second term (since the first
is already diagonalized). We know from previous discussions we know that the second
term is an involution for a unit vector $\hat{b}$ which gives eigenvalues of
$\frac{1}{2} \pm \abs{\vec{b}} \ge 0$. Which implies that $\vec{b}$ must have
a magnitude of less than or equal to $\frac{1}{2}$. We can saturate this
inequality to get a pure state with the density matrix
$\rho = \Ket{\vec{d},+}\Bra{\vec{d},+}$ where $\vec{d}$ is a scaled version of $\vec{b}$.

\section{New Postulates of QM}
We can now rewrite our fundamental postulates of QM as follows,
\begin{enumerate}
    \item A physical state is a density operator $\rho$
    \item Observables are Hermitian operators
    \item The expecatation of an observable is $\left[A\right] = \tr \left(\rho A\right)$
    \item The probability to measure $\lambda$ from $A$ to be
        $\tr \left(\rho P_{\lambda}\right)$. Where $P_{\lambda}$ is a projection
        operator into the eigenstate with eigenvalue $\lambda$.
\end{enumerate}
A new problem now is what happens when the resulting density operator is the same
when combining two different setups. I.E. if you compare two preparations which have
the same density matrix how can you distiguish them? In this formalism you can't!
Once things are prepared you cannot distinguish between the two setups as there is
a "many-to-one" relationship between the initial ensembles and the final
density matrix.

\begin{theorem}[Unknown]
\label{unknown}
Let $\left\{\Ket{\phi_i}\right\}_{i=1}^{n} \subset \mathcal{H}$ and
$\left\{\Ket{\psi_i}\right\}_{i=1}^{n} \subset \mathcal{H}$, and we have
the following condition,
$\sum_{i=1}^{n} \Ket{\phi_i}\Bra{\phi_i} = 
\sum_{i=1}^{n} \Ket{\psi_i}\Bra{\psi_i}$ iff
$\exists U \in U(n)$ s.t. 
$\Ket{\psi_k} = \sum_{j=1}^{n} U_{jk} \Ket{\phi_j}$
\end{theorem}

We can apply this theorem to density matrices using the following setup
\begin{corollary}
Two pure state $\Ket{\phi}$ and $\Ket{\psi}$ have the same
density matrix if $\Ket{\phi} \propto \Ket{\psi}$. In other words
if two pure states have the same density matrix then they will represent
the same quantum state of the system. The only ambiguity is the phase
\end{corollary}

\begin{corollary}
Mixture 1 with $\left\{\left(\sqrt{p_i}, \Ket{\psi_i}\right)\right\}$
and mixture 2 with $\left\{\left(\sqrt{q_i}, \Ket{\phi_i}\right)\right\}$
have the same $\rho$ iff $\exists U \in U(n)$, s.t.
$\sqrt{p_i} \Ket{\psi_i} = \sqrt{q_i} \Ket{\phi_i} U$ (implicitly
the states are in a vector structure).
\end{corollary}

