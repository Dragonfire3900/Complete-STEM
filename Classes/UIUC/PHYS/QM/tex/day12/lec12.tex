\section{Chapter 3 Sakurai. Symmetry}
This whole chapter is about spin and rotation via symmetry. He typically connects it to
group theory quite a bit which is nice! This in theory should be self contained so let's go!

\section{Group Theory}
Def: A set $G$ with a binary operation $*: G \times G \rightarrow$ st the following is true
\begin{enumerate}
    \item $*$ is associative
    \item $\exists e \in G$, st $\forall g \in G$, $e * g = g = g * e$ (identity)
    \item $\forall g \in G$, $\exists g^{-1} \in G$, st $g^{-1} * g = e = g * g^{-1}$
\end{enumerate}
A good example of a group are the integers which are a group under addition but not
multiplication ($<\mathbb{Z}, +>$). Another definition is abelian

Def: A group $G$ is abelian if $*$ is commutative. 
We denote $*$ as $+$ and $e$ as 0 in this case. Otherwise, $G$ is called non-abelian.

Most of the symmetry groups we will deal with are going to deal with are non-abelian.
We'd like to organize the systems we can solve into symmetry structures. This can be done
through representations which will be explained later.

Rotation will be shown to be $SU(2)$ instead of $SO(3)$ which actually allows for Fermions
to exist somehow.

Def: A Lie group $G$ is a group that is also a differentiable manifold w/ $*$ that
is smooth. I.E. you can apply calculus in some way. 

We're going to work through this largely by showing examples instead of working
with the formal definitions. So pay attention. Importantly $SO(3)$ is a projective manifold
of $\mathbb{R}^3$. It is a projection of the $SU(2)$ space. 

Def: A field $\left<\mathbb{F}, +, \times\right>$ is a set that is an 
abelian group under $+$ \& $\mathbb{F}^* \equiv \mathbb{F} \ \{0\}$ is a set that is an
abelian group under $\times$ \& $\forall f, g, h \in \mathbb{F}$,
$f \times \left(g + h\right) = f \times g + f \times h$.

Def: The simplest non-abelian group is $GL(n, \mathbb{F}) = \{ n\times n 
\text{invertible matrices over } \mathbb{F}\}$ is a group.

The idea is that given
some group how do we make sense of the elements of the group. We can do this by understanding
the action of the group on some space. We can find a representation by finding the element
in another group that we do understand

Def: $\phi: G_1 \rightarrow G_2$ is called a homomorphism $\forall f, g \in G$, if
$\phi \left(f * g\right) = \phi(f) * \phi(g)$. 
A bijective homomorphism is called an isomorphism.

\subsection{$\mathbb{Z}_n$ group}
This group is defined as $\left<\mathbb{Z}_n, +\right>$, $\mathbb{Z}_n = \{0, \dots, n-1 \}$.
With the operation $a + b (\mod n)$. is an abelian group. If $p$ is a prime, then
$\left<\mathbb{Z}_p, +, \times\right>$ is a field. The claim is that $\mathbb{Z}_2$ is 
isomorphic to something similar to the Pauli matrices
$$
    \mathbb{Z}_2 \equiv \{0, 1 \} \cong
    \left<\left\{I, \sigma_x\right\}, \times\right>
$$
It's much easier to work with $\mathbb{Z}_2$ rather than the Pauli matrices.

Def: A representation of $G$ is a homomorphism $\phi G \rightarrow GL(n, \mathbb{C})$. This
is a particular definition is for this class in particular. The elements of the quantum state
will be elements of the $GL$ space. We can reduce these representations if we can block
diagonalize these representations into independent parts.
$$
    \mathcal{H} = \oplus_{n} \mathcal{H}_n
$$

Def: Vector spaces $V$, $U$, $W$, $U = V \oplus W$ st., 
$\forall u \in U$, $\exists v \in V \& w \in W$ st. $u = v + w$. $U \cap W = \{0\}$.

\subsection{Identity proofs}
Suppose $e_1$ \& $e_2$ are identities of $G$. Then
$e_1 = e_1 * e_2 = e_2$.
Which shows that the identity elements are unique.

Suppose $f, h$ are inverse of $g \in G$. Then,
$f = f * e = f * (g * h) = (f * g) * h = e * h = h$
which shows that inverses are unique

\subsection{Symmetric groups}
Def: $S_n = \{ \text{all permutations of } n \text{ distinct elements}\}$. This is non-abelian
for $n > 2$. Importantly $\abs*{S_N} = N!$. A good example is
$S_2 \cong \left< \mathbb{Z}_2, +\right>$. Another good example is $S_3$ which shows that
the individual permutations can be broken down into a product of transpositions. The evenness
is preserved under any representation. Any even number of transpositions will always be this way.
From this we can give the following definitions
\begin{enumerate}
    \item even permutation: $\pi$ is a product of even \# of transpositions
    \item odd permutation: $\pi$ is a product of odd \# of transpositions
\end{enumerate}
We will encounter identitcal paritcles in 581 a lot more. 

{\color{green} A few exercises for you. Show the two following $\phi(e_1) = e_2$ and
$\phi(g^{-1}) = \phi(g)^{-1}$}

\section{Lorentz Group}
In order to understand the applications of group theory correctly we need to review
how special relativity plays into things. This means we need to deal with the Lorentz group.
We need to introduct a scalar product on $\mathbb{R}^4$.

Def: Minkowski scalar product: 
$\left<\;,\;\right>_M: \mathbb{R}^4 \times \mathbb{R}^4 \rightarrow \mathbb{R}$ 
a bilinear map s.t.
$\forall x, y \in \mathbb{R}^4$, 
$\left<x, y\right>_M = \left<x, gy\right>_{\text{Euclidian}}$, where $g$ is the
Minkowski "metric" (it's not really a metric). In particular,
$x_{\mu}^2 = x^t g x = x_0^2 - \vec{x}^2$ 
$\left(x = \begin{pmatrix} x_0 \\ \vec{x} \end{pmatrix} \in \mathbb{R}^4, \;x_0 = ct\right)$
When you demand that the Minkowski norm is constant you end up finding a hyperboloid. There
are two different kinds of seperations from the origin, space-like and time-like. The question
then becomes {\color{red} what transformations keep the Minkowski norm fixed?}.

Def: Lorentz group $O(1, 3) \subset GL(4, \mathbb{R})$ s.t.

$$
  O(1, 3) = \{ \Lambda \in GL(4, \mathbb{R}) \vert
  \left<\Lambda x, \Lambda y\right>_M = \left<x, y\right>_M \forall x, y \in \mathbb{R}^4\}  
$$

Effectively $\Lambda^t g \Lambda = g$. The first main property is that
$\det \left(\Lambda\right) = \pm 1$. This property implies that the general Lorentz transformation
can be broken into two parts
$$
    O(1, 3) = SO(1, 3) \sqcup O(1, 3)_{-}
$$
The $SO(1, 3)$ is considered the "proper Lorentz transformation" while the other
are the "improper Lorentz transforms". Furthermore $SO(1, 3)$ can be broken down
into the $SO(1, 3)^{+}$ (orthochronos/ restricted Lorentz group) and the 
$SO(1, 3)^{-}$ (time reversal group). Only the $SO(1, 3)^{+}$ is a group as it contains
the identity. Looking at the dimensionality of the various groups we are working with
reveals a subgroup of $SO(1, 3)^{+}$ being $SO(3)$. This is done by considering the block
matrix $\begin{pmatrix} 1 & 0 \\0 & R \end{pmatrix}$ with $R \in \mathbb{R}^{3 \times 3}$.

Importantly $SO(3)$ is a classical rotation in $\mathbb{R}^3$. This is from the following
Theorem.

ThM: Any $R \in SO(3)$ is a rotation around some axis. The overall proof is as follows,
$\det (R) = 1 = \lambda_1 \lambda_2 \lambda_3$, $\lambda_i$ are the eigenvalues of $R$.
Taking the complex conjugate of the e.value equation we can conclude that
$\bar{\lambda}_i$ is also an e.val. We can also show that the norm of some eigen vector
$\nu$ is preserved under the action of $R$ which implies that the eigen values are on the
unit circle. As a result the e.vals come in complex conjugate pairs and there must always be
one axis of rotation with an e.val of $1$. 

Importantly the determinant is continuous and the top vs. bottom rows are disconnected due
to different signs in the determinant. We can also show that the two columns are disconnected
from one another.
This can be established by looking at the general form
of the $0, 0$ element. 

\subsection{Boosts}
Def: $\Lambda \in SO(1, 3)^{+}$ is called a boost in the direction
$\vec{n} \in \mathbb{R}^3$ if the plane orthogonal to $\vec{n}$ is invariant
under $\Lambda$. Boosting can be written as the following
$$
\begin{pmatrix}
    \cosh \chi & \sinh \chi & 0 & 0 \\
    \sinh \chi & \cosh \chi & 0 & 0 \\
    0 & 0 & 1 & 0 \\
    0 & 0 & 0 & 1
\end{pmatrix}
$$
where $\cosh \chi = \frac{1}{\sqrt{1 - \beta^2}}$. In this course we will be taking the 
active transformations instead of the passive transformations. 
