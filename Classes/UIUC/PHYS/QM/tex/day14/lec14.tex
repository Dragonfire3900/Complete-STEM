\section{Review of last time}
Given a rotation $R_{\vec{n}}(\theta)$ around $\vec{n}$ by angle $\theta$
$$
    R_{\vec{n}} \longleftrightarrow^{1:2} U_{\vec{n}}(\theta) =
    e^{-\frac{i}{2}\theta \vec{n}\cdot\vec{\sigma}}
$$
s.t.
$$
    U_{\vec{n}}(\theta) \left(\vec{x}\cdot\vec{\sigma}\right)
    U_{\vec{n}}(\theta)^\dagger = 
    \left(R_{\vec{n}}(\theta) \vec{x}\right)\cdot \vec{\sigma}
$$
Given a boost along $\vec{n}$ by $\frac{v}{c} = \tanh \chi$,
$$
B_{\vec{n}}(\chi) \longleftrightarrow^{1:2} H_{\vec{n}}(\chi) = 
e^{\frac{1}{2} \chi \vec{n}\cdot\vec{\sigma}}
$$
s.t.
$$
H_{\vec{n}}(\chi) x_{\mu} \cdot \sigma_{\mu} H_{\vec{n}}(\chi) =
    \left(B_{\vec{n}}(\chi) x_{\mu}\right) \cdot \sigma_{\mu}
$$

\section{Rotations}
These two constructions show that there is a two to one cover between
$SO(1,3)^+$ and $SL(2, \mathbb{C})$. Since we are not going through QFT
we will restrict ourselves to the rotation subgroup $SO(3)$.

Def: Homomorphism. For $\phi: G_1 \rightarrow G_2$ is a homomorphism if
$\forall g_1, g_2 \in G$, $\phi(g_1, g_2) = \phi(g_1) \phi(g_2)$.

It's hard to understand the action of a group so we can fix
something that helps us visualize this. We can do this by
fixing a vector space and see how the group acts
on that matrix space. The typical way we can do this
is by mapping from the original group to $GL(n, \mathbb{C})$
via a homomorphism

Def: A representation of a Lie group $G$ is a continuous
group homomorphism $\phi: G \rightarrow GL(n, \mathbb{C})$

The question is can we decompose this representation into
multiple simplier representations? As good example is the tensor
product between two vector spaces for spins. The simplest possible
space is the irreducible representation of a vector space.

Def: $W \subset \mathbb{C}^n$ is called invariant if $P(G)W \subset W$. Thus,
$W$ is an invariant subspace of $\mathbb{C}^n$

Def: A rep $P$ is called irreducible (irrep) if
$\{0\}$ is the only invariant subspace.

So if you scan through actions of all elements of $G$ and there isn't an
invariant subspace for all actions.
We will be showing how to construct irreps for the various groups. 

\section{Irreps?}
In QM it's important to realize that these irreps are our friends and are very helpful
for many different problems. In particular we are interested in irreps
which are in the unitary subgroup of $GL$.

If $P$ is a unitary rep, i.e. $P(G) \subset U(n)$ then
$$
    P(g) =
    \begin{pmatrix}
        A & 0 & 0 \\
        0 & B & 0 \\
        0 & 0 & C
    \end{pmatrix}
$$
In other words it is block diagonal.

Let $W \subset \mathbb{C}^n$ be an invariant subspace of $P$
then $\forall w \in W$, $w^\perp \in W^\perp$, $g \in G$, then
$\left<P(g) w, P(g) w^\perp\right> = \left<w, w^\perp\right> = 0$.
This comes from a unitary representation not affecting the 
inner product of the space. Since $w$ and $w^\perp$ have an inner product
of $0$, We can break $\mathbb{C}^n$ into two different subspaces using the
direct sum. I.e. $\mathbb{C}^n = W \oplus W^\perp$.


\subsection{Irreps of $SO(3)$}
Let $R \in SO(3)$ that is constant. Let's $\vec{y} = R \vec{x}$ and we will look
at how the gradient transforms. From chain rule $\nabla \vec{y} = R \nabla\vec{x}$.
From this we can figure out how the Laplace operator transforms,
$\nabla^2_y = \nabla_y \cdot \nabla_y = \nabla_x R^t R \nabla_x = \nabla^2_x$.
Thus, the Laplace operator is invariant under $SO(3)$.

Let $\mathcal{H}_l = \{ \text{deg } l \text{ homogeneous harmonic polynomial in } x, y, z
\text{ over } \mathbb{C}^n\}$
Putting in these polynomials into the Laplace operator,
$\nabla^2_y f(x) = 0 = \nabla_x^2 f(R \vec{x})$. Thus, if you
rotate the polynomials $f$ you get another $f$ with the same
overall setup. As a result we would like to see how these various polynomials
rotate between each other under $R$. We will end up finding that the dim of
... is 12 + 1

Let $f \in \mathcal{H}_l$ be expanded as
$$
f(\vec{x}) = \sum_{\alpha \ge 0, \beta \ge 0, l - \alpha - \beta \ge 0}
(x + iy)^\alpha (x - iy)^\beta z^{l - \alpha - \beta}
$$ 
Let's switch over to
spherical coordinates, we will define $\theta$ as the polar angle and $\phi$ as the azimuthal
angle. We get the following
$$
    f(\vec{x}) = \sum_{\alpha, \beta}
    r^l \left(\sin \theta\right)^{\alpha+\beta}
    \left(\cos \theta\right)^{l - \alpha - \beta}
    e^{i \varphi(\alpha - \beta)}
$$
From here we will define $m \equiv \alpha - \beta$ which we can sum over.
We can change our sum to have this difference be fixed. We get the following
$$
= \sum_{m = -l}^{l} r^l \sum_{\alpha, \beta, \alpha - \beta = m}
    A_m g^m_l \left(\theta\right) e^{im \varphi}
$$
We are going to demand that this whole polynomial to be harmonic. Which amounts to
solving for the coefficients $A_m$. Using the Laplace operator we can find the
differential equation which $g_l^m$ must satisfy. This turns out to the differential
equation to the associated Legendre polynomials. I.E.
$g_l^m(\theta) = P_l^m(\cos \theta)$. We can then find the normalization coefficients
of this wave function over the angular component. We end up with the spherical harmonics
$$
    Y_l^m(\theta, \phi) = (-1)^m 
    \sqrt{\frac{2l + 1}{4 \pi} \frac{(l-m)!}{(l+m)!}}
    P_l^m(\cos \theta) e^{im \phi}
$$
Which implies that $\dim \mathcal{H}_l = 2l + 1$. Using this normalization
we also have an orthogonality relationship over the solid angles. In particular
$$
\int Y_l^m Y_{l'}^{m'} d \Omega = \delta_{mm'} \delta_{ll'}
$$
It's this orthonormalization condition which gives the normalization coefficient.
We can now write down the action of the rotation on these polynomials! Just like we
defined previously we get the following
$$
    P^{(l)}(R^{-1}) Y_l^m(\vec{n}) \equiv
    Y_{l}^m(R \vec{n}) = 
    \sum_{-l}^{l} P_{m'm} (R^{-1}) Y_l^{m'}(\vec{n})
$$
We can show that the inverse is needed to have a homomorphism. Recall also
that $R^{-1} = R^t$. We can use this to find the coefficients of the rotation
matrix by expanding two different rotations.

It's much harder to show that these are irreps of $SO(3)$ but we can just trust for now.
Importantly only $Y_l^0(\hat{z}) \ne 0$. If $W \subset \mathcal{H}_l$ is invariant
then take $f\ne 0 \in W$ which implies
$\exists \vec{x}$ s.t. $f(\vec{x}) \ne 0$, then $\exists R$ s.t. $R\vec{x} = \hat{z}$.
Thus, $f(R^{-1} \vec{x})$ doesn't vanish at $\hat{z}$ which is a contradiction. Thus,
we have shown that $Y_l^0$ is irrep?

\section{Double cover implications}
Remember that we have a double cover between $SO(3)$ and $SU(2)$.

Through a bit of work we can show that for $SU(2)$ we only need two complex
numbers to represent any element. The main constraint that we have is the
determinant being 1 (which implies that their magnitude squared is one). Thus, 
we are dealing with a three sphere in $\mathbb{R}^4$. Effectively, we end up
projecting $A$ and $-A$ to the same rotation. As a result, we need to mod
out the two identity elements to get $SO(3)$, in other words a projective 
representation $\mathbb{R}P^3 = SO(3) = S^3 / \mathbb{Z}_2$. Thus,
we have the following representation
$P: SO(3) \rightarrow PGL(n, \mathbb{C}) = GL(n, \mathbb{C}) / \mathbb{Z}_2$.
Modding this out works by "identifying" $\pm A$ (meaning that they are equivalent).

We can end up considering $SU(2)$ and the quantum rotation group and $SO(3)$ the classical
rotation group

\section{Irreps of $SU(2)$}
The fundamental rep of $SU(2)$ is 
$$
    P
    \begin{pmatrix}
        a & b \\
        -\bar{b} & \bar{a}
    \end{pmatrix} = 
    \begin{pmatrix}
        a & b \\
        -\bar{b} & \bar{a}
    \end{pmatrix}
$$
Given this representation we can extend our definition of $\mathcal{H}$ to QM.
Let
$V_n = \{ \text{deg } n \text{ homogenous polynomials in } z_1, z_2 \text{ over } \mathbb{C} \}$.
We will end up $\dim V_n = n + 1$. For all odd dimensional representations we can find
a corresponding element of $\mathcal{H}_l$ which matches up to it. These are the integer
spin particles. However, for odd $n$ we will end up finding new representations
which have half integer spins. We can use the various monomials of $z_1$ and $z_2$ as the
basis for this representation. I has the following inner product which will give good
normalization
$$
    \left<
        \sum_{k=0}^n a_k z_1^k z_2^{n-k},
        \sum_{m=0}^n b_m z_1^m z_2^{n-m}
    \right> = \sum_{k=0}^n k! (n-k)! \bar{a}_k b_k
$$
With the orthonormal basis
$$
    \vec{\phi}_k \begin{pmatrix} z_1 \\ z_2 \end{pmatrix} =
    \frac{z_1^k z_2^{n-k}}{\sqrt{k! (n-k)!}}
$$
With the following rotational action
$$
    P(A) \phi_k(z) = \phi_k(A^t z) = 
    \sum_{i=0}^n P(A)_{ik} \phi_i (z)
$$

\subsection{Euler Angle Example}
We will be going through a demo of the Euler angles to see how this works.
To start, recall $R \in SO(3)$ can be decomposed into three different rotations
$R = R_z(\alpha) R_y(\beta) R_z (\gamma)$. These various angles are how much the
body attached axis changed relative to a fixed axis. There's some mathematica code
which will help you visualize these angles

\subsection{Detecting the negative sign on a full rotation}
This is the belt example from Penrose.

Start with a polarized thermal neutron. Split the beam in two ways,
one path with continue straight while the other will be offset by 90 degrees.
Put two reflectors along these paths such that they will collide into another beam
splitter.
On the upper path put a magnetic field to introduce a phase difference (say
by $B_0 \hat{z}$). If you tune your magnetic field correctly you can get
interference between the two paths. We specifically get the following
$$
    \mathcal{H}_{\text{int}} = -\vec{\mu} \cdot \vec{B} =
    - \frac{g_n \abs*{e} \hbar}{2 m_p c} \frac{S_z}{\hbar} B_0
$$
Under this interaction we will get the following phase evolution
$$
    e^{-i \mathcal{H} T} \Ket{\pm} =
    e^{\frac{i}{\hbar} \frac{g_n \abs*{e} B_0 T}{2m_p c} S_z} \Ket{\pm} =
    e^{\mp i \omega \frac{T}{2}} \Ket{\pm}
$$
When rotating only by $2 \pi$ you will only end up rotating around phase wise
by $\pi$ which will pick up a negative sign. Thus, we can show that we have
half integer spins. 
