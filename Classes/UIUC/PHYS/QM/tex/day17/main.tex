\documentclass[12pt]{article}

\usepackage{tikz}

\usepackage{mdframed}
\usepackage{amsmath}
\usepackage{amssymb}
\usepackage{amsfonts}
\usepackage{amsbsy}
\usepackage{amsthm}
\usepackage[nomath]{stix}
\usepackage{bbm}
\usepackage{comment}
\usepackage{braket}
\usepackage{mathtools}
\usepackage{esdiff}

\newcounter{hwProblem}
\newcounter{hwPart}[hwProblem]
\setcounter{hwProblem}{1}

\providetoggle{showPrompts}
\settoggle{showPrompts}{true}

\mdfdefinestyle{hwPromptBox}{linecolor=blue}

\newenvironment{hwProblem}[2][-1]{
    \ifnum #1>0
        \section*{Problem \arabic{hwProblem} ({\it #1 pts.}): #2}
    \else
        \section*{Problem \arabic{hwProblem}: #2}
    \fi
    \stepcounter{hwProblem}

    %Defining what a problem part is
    \newenvironment{hwPart}[2][-1]{
        \stepcounter{hwPart}
        \ifnum##1>0
        \subsubsection*{Part \Alph{hwPart} ({\it ##1 points}): ##2}
        \else
        \subsubsection*{Part \Alph{hwPart}: ##2}
        \fi

    }{}
    
    \iftoggle{showPrompts}{
        \newenvironment{hwPrompt}{
            \begin{mdframed}[style=hwPromptBox]
        }{\end{mdframed}}
    }{
        \excludecomment{hwPrompt}
    }
}{}

\DeclarePairedDelimiter{\abs}{\lvert}{\rvert}
\DeclareMathOperator{\tr}{tr}

\newtheorem{theorem}{Theorem}[section]
\newtheorem{corollary}{Corollary}[theorem]
\newtheorem{lemma}[theorem]{Lemma}

\theoremstyle{definition}
\newtheorem{definition}{Definition}[section]

\includecomment{pbStatement}
\includecomment{longform}

\title{Lecture Day \#17}
\author{Joel Kelsey}
\date{10-22-2025}

\begin{document}
\maketitle

\section{Irreducible proofs}
Let's set the definition of
$$
    V_j = \text{span}\left\{ \Ket{jm} \right\}_{m=-j}^j
$$
The central question for now is
"Why is this space irreducible under $so(2)$ action?"
This likely is to extend the homework problems and the spherical
harmonics being irreducible representations of $SO(3)$.

"In graduate school you should question what you learn" since none of us
thought about this question before. The technique
he's going to talk about is generalizable for any highest weight
representation.

\subsection{Example technique}
Suppose that $V_j = V \oplus W$. 
\begin{definition}[Direct Sum]
\label{directSum}
"Any vector in the total space can be written as the sum of two vectors from the
various subspaces". Both of these must only intersect at the origin
\end{definition}

\begin{definition}[Quotient Space]
\label{quotientSpace}
This is denoted by $\mathbb{R}^2 / \mathbb{R}$ being the quotient
space between $\mathbb{R}^2$ and $\mathbb{R}$ or $\mathbb{R}$ is modded out.
This idea also needs the idea of equivalence classes. In general though,
the quotient space means that $a \sim b$ if $a-b \in W$ are equivalent
in the space $V/W$.
\end{definition}

\begin{definition}[Equivalence Class]
\label{equivalenceClass}
There are three different properties that the equivalence classes must satisfy
\begin{enumerate}
    \item $a \sim a$, meaning $a$ is related to itself
    \item $a \sim b \Leftrightarrow b \sim a$
    \item $a \sim b$, $b \sim c$ implies $a \sim c$, transitive properties
\end{enumerate}
\end{definition}

Continuing on from these definitions, lets say $u = \sum_m a_m \Ket{jm} \in V$ and we can
order these according to the largest $m$ that has a non-zero coefficient, denoted $M$.
We're going to act on this state with $J_-$. By definition this will lower $m$ by one
and remove one of the states. We can then operate on this 
$J_-^{M+j}$ times which will give you the $\Ket{j -j}$ state. Similarly you can show that
$W$ must also contain this state which is \underline{not the origin}. Thus, these
two spaces are related and thus, not indept. From this $V_j$ must be invariant.

\section{Density matrix}
Recall the definition of the density matrix $\rho$ as a 
Hermitian, positive semi-definite operator $\rho: \mathcal{H} \rightarrow \mathcal{H}$,
s.t. $\tr(\rho) = 1$

\begin{definition}[Incoherent density matrix]
\label{incoherentDM}
Let $\left\{\Ket{\phi_n}\right\}$ be an o.n. basis of $\mathcal{H}$.
If $\rho = \sum_{k=1}^{N} w_k \Ket{\phi_{i_k}} \Bra{\phi_{i_k}}$, then
we say $\rho$ is \underline{incoherent}, in this basis. Otherwise
we say $\rho$ is coherent.
\end{definition}

This coherence is \underline{not the same} as the coherent state 
from the simple harmonic oscillator. In this case coherence
is referring to the relative phase between the two states being fixed and known.

Let's say that
$$
\Ket{\psi} = \sum_n \abs*{c_N} e^{i \phi_n} \Ket{\psi_n}
$$
and let's do an experiment of how the density matrix will look with this
setup
$$
    \rho = \Ket{\psi} \Bra{\psi} = \sum_{n,m}
        \abs*{c_n}\abs*{c_m} e^{i (\phi_n - \phi_m)}
        \Ket{\psi_n} \Bra{\psi_m}
$$
Say that this prepared state had some interactions with the environment.
In this case the phase factor between the states will be shifted. As a result
the phase will average out to zero in an experiment. This means that for experimental
setups the off diagonal elements will end up disappearing. This is known as a
\underline{decoherence} problem and comes up quite a bit in QIS. This can
also happen if the energy states have a little bit of uncertainty over time on them

We also mentioned on monday that there is a many to one relationship between the setup
of an experiment and the density matrix. Such as
the density state:
$$
    \rho = \frac{1}{2} \Ket{+}\Bra{+} + \frac{1}{2} \Ket{-}\Bra{-}
$$
can have a unitary transform applied to it without affecting the density matrix.
Similarly we can have an isotropic distribution of spins which will
result in the exact same density matrix

\subsection{Von Neumann Entropy}
This is the "entropy" of your density matrix. We will be denoting
entropy as $S(\rho)$. You can think of this entropy in a typical
way of "disorder". However, you can also view this
as "how much information you can gain by doing an experiment".

With a high certainty prediction, "the sun will rise", an experiment
will not give you much information by doing that experiment. Effectively
common events do not give you more information. As a result you'd
expect a measurement's information density to be additive and
be related to $1/\mathbb{P}$. We can build the additive property
by using a logarithm. We can use this to construct an average amount of information
given using Shannon Entropy
\begin{definition}[Shannon Entropy]
\label{shannon}
This is defined by
$$
    \text{expected information} = -\sum_i \mathbb{P}_i \log \mathbb{P}_i
$$
where we are summing over all possible outcomes
\end{definition}

We can build the Von Neumann Entropy related to the Shannon entropy but
accounting for the mixture of the system.
\begin{definition}[Von Neumann Entropy]
\label{vonNeumann}
This is heavily related to Shannon Entropy but has the definition from
the density matrix $\rho$ of
$$
    S(\rho) = - \tr \left(\rho \log \rho\right) =
    - \sum_k D_{kk} \log D_{kk}
$$
where $\rho = V D V^\dagger$. This relies on the density matrix being diagonalizable
\end{definition}
In the problem set we need to be careful in construction of the
density matrix because we don't have an orthonormal set. So be careful

\begin{corollary}
$S(\rho) = 0$ iff $\rho$ is pure. This is relatively straight forward
from one eigenvalue being non-zero and the rest being zero. As implies
from the pure state.
\end{corollary}

Suppose that $\left\{\Ket{\psi_n}\right\}$ is an orthognormal set.
If $\rho = \sum_{n=1}^{N} w_n \Ket{\psi_n} \Bra{\psi_n}$, and
$\sum_{n=1}^N w_n = 1$. For which 
$\left\{w_N\right\}$, is $S(\rho$ maximized? It will end up being
the uniform distribution! We can show this using Lagrangian mechanics
and some derivatives. 

\subsection{Evolution of Density Matrices}
People call the density matrix an "operator" but it's not really
an observable of the Hilbert space. The reason is how it evolves in time.
We can use the unitary time evolution operator. In the Schr\"odinger
picture we can just evolve the states getting
$\rho(t) = U(t) \Ket{\psi, 0}\Bra{\psi, 0} U(t)^\dagger$
but in the Heisenberg picture
$\rho = U(t) \rho U^\dagger(t)$. But this doesn't match
up with what we're getting from the Schr\"odinger picture.
As a result the density operator \underline{does not evolve in 
the Heisenberg picture}. In other words
$\rho_H (t) = \Ket{\psi, 0}\Bra{\psi, 0}$.

As a result the density matrix commutes with the
Hamiltonian and we can simultaneously diagonalize the density
matrix and the Hamiltonian. So we can diagonaliz $\rho$ in the
energy eigenbasis. This builds up to quantum statistical mechanics!
The question we now want to answer is "what does this matrix
look like".

One way we can do this is by fixing the mean energy of the
system $\mathcal{E} = \left[H\right] = \tr \left(\rho H\right)$
and see how the entropy can be maximized.
\begin{gather*}
    \mathcal{L} = - \tr \left(\rho \log \rho\right) +
        \alpha \left(\rho_{kk} - 1\right) -
        \beta \left(\tr \left(\rho H\right) - \mathcal{E}\right) \\
    \diff{{\mathcal{L}}}{{\rho_{kk}}} = - \log \rho_{kk} - 1 
        + \alpha - \beta E_k = 0 \\\Downarrow\\
    \rho_{kk} = e^{\alpha - 1 - \beta E_k} \\\Downarrow\\
    \rho_{kk} = \frac{e^{-\beta E_k}}{\sum_k e^{-\beta E_k}}
\end{gather*}
Overall the result we can show that the form of the density matrix is
$$
    \rho = \frac{e^{-\beta H}}{\tr \left(e^{-\beta H}\right)}
$$

\subsection{Central Force Problems}
We need this kind of formalism in order to discuss the Hydrogen atoms problem in the
homework set. A few weeks ago we derived the angular momentum to be the
same as the classical definition, $\vec{L} = \vec{x} \times \vec{p}$. Using this
we can get the following for the $L_z$ operator.
$L_z = -i \hbar \left(x \diffp{}{y} - y \diffp{}{x}\right) = -i \hbar \diffp{}{\phi}$.
We would like to use spherical coordinates to break the radial components from the
angular components. Which better match our problems. Similarly
$$
    L^2 = \vec{L} \cdot \vec{L} = -\hbar^2
        \left[\frac{1}{\sin \theta} \diffp{}{\theta} \sin \theta \diffp{}{\theta}
        + \frac{1}{\sin^2 \theta} \diffp[2]{}{\phi}\right]    
$$
Recall that in the Hamiltonian we have $-\nabla^2$ which will give
an angular momentum barriers to areas close to the origin. This is
why the radial solutions tend to be "pushed" away from the origin.

In general with a very heavy nucleus we can model the Hammiltonian
using a purely radial potential. I.E. 
$H = \frac{\vec{p}^2}{2m} + V(r)$, and $r = \sqrt{\vec{x}^2}$. In
order to find the spectrum we need to look at the symmetry of the problem.
We can do this by looking at the commutators with the generators of
rotation, $L$. Thus,
\begin{gather*}
    \left[\vec{L}, \vec{p}^2\right] = \left[\vec{x}\times\vec{p}, \vec{p}^2\right] \\
    = \left[\left(\vec{x} \vec{p}^2\right) - \vec{p}^2 \vec{x}\right] \times \vec{p} \\
    = \left(\left[\vec{x}, \vec{p}^2\right]\right) \times \vec{p} \\
    = i \hbar \vec{p} \times \vec{p} =0
\end{gather*}
Similarly $\left[\vec{L}, \vec{x}^2\right] = 0 \Rightarrow \left[\vec{L}, f(r)\right] = 0$.
Thus, we have a symmetry in $\left[\vec{L}, H\right] = 0$. Even though the entire
vector does commute, the individual components do not commute. So we need to choose
a "maximally commuting set" which we will simultaneously diagonalize. By convension
we are going to choose $L_z$, $L^2$, and $H$. We choose $L_z$ because it has
a simple form when written as a differential equation. We can then
seperate the differential equation in to the radial and angular components. In
particular we know that the radial solutions should be the spherical harmonics since
they were constructed to be the eigenbasis of $L_z$ and $L^2$.
Then we can have $\Psi(\vec{x}) = R_n(r) Y_l^m(\theta, \phi)$.

At this point this reduces to a classical mechanics effective potential problem
in the radial case. We can use a specific potential which for the hydrogen atom
this if $-Z e^2 /r$. Importantly we can do a type of asymptotic analysis to give
us a direction for what kinds of solutions would work. 

In the case $\abs*{V} \ll r^{-2}$
as $r \to 0$, then there are two solutions
$U(r) \propto A r^{l+1} + B r^{-l}$. This comes from the potential term being
dropped as it is small relative to the potential barrier at the origin

In the opposite case, as $r \to \infty$,
if $V(r) \to 0$, then the potential terms are small relative to $E_n$.
In this case we end up with solutions along the lines of
$U \propto A e^{-\kappa r} + B e^{\kappa r}$. Whever $\kappa = - \frac{2m E_n}{\hbar^2} > 0$.
We can then redefine $U = \eta^{l + 1} e^{-\eta /2} \omega(\eta)$ where
$\eta = 2 \kappa r$.

\end{document}
