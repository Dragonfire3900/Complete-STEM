\section{Feynman Path Integrals}
Recall that the Feynman kernel is
$$
    K(x, t;y, t_0) = \int \mathcal{D}\left[x\right] e^{\frac{i}{\hbar} S(\vec{x})}
$$
Today we are wanting to focus on how to do these calculations explicitly. Recall
that $\mathcal{D}$ is some kind of measure on the path space. This allows us to
compare how significant certain paths are. Remember that $S$ is also the 
action with the constraints $\vec{x}(t_0) = y$ and $\vec{x}(t) = x$.

For this formalism there are three different assumptions we need to make
\begin{itemize}
    \item In the limit $\Delta t \to 0$, $K(x_{n+1}, t_{n+1}; x_n, t_n) 
        \sim e^{\frac{i}{\hbar} S}$.
        This comes from approximating the action as a straight path between the two points.
        Fomr here you could approximate the action using just the lagrangian itself.
    \item Straight lines for sufficiently smalle $t$ can be used to approximate the action.
        This implies $\mathcal{L} \approx \left[
            \frac{1}{2} m \left(\frac{x_{n+1} - x_n}{\Delta t}\right)^2 -
            V \left(\frac{x_n + x_{n+1}}{2}\right)
        \right] \Delta t$
    \item $K(x_{n+1}, t_{n+1}; x_n, t_n) = \delta (x_{n+1} - x_n)$. This helps pick the
        normalization of the actions
\end{itemize}
This is also similar to the WKB approximation using the classical path as a minimum or
stable path which will reproduce classical physics.

The question is now, How do we get to the kernel definition above? Recall
$\mathcal{L} = \frac{1}{2} m \dot{x}^2 - V(x)$ with the corresponding
Hamiltonian $H = \frac{1}{2} \frac{p^2}{m} + V(x)$. In QM once we have a
sufficiently simple Hamiltonian we know the time evolution operator
$U(t_{n+1}, t_n) = e^{-\frac{i}{\hbar} H \Delta t}$. We can apply this
operator multiple times to get to the two different time periods,
$U(t, t_0) = \left(e^{-\frac{i}{\hbar} H \Delta t}\right)^{N+1}$
\begin{align*}
    K(x, t, y, t_0) 
    &= \Bra{x} U(t, t_0) \Ket{y} \\
    &= \int \Bra{x} e^{-\frac{i}{\hbar} H \Delta t} \Ket{x_n}
        \Bra{x_n} e^{-\frac{i}{\hbar} H \Delta t} \Ket{x_{N-1}} \dots
        \Ket{y} dx_1 \dots dx_N \\
    &= \dots \Bra{x_{n+1}} e^{-\frac{i}{\hbar} \left(\frac{p^2}{2m} + V\right) \Delta t}
        \Ket{x_n} \\
    &\approx
        \Bra{x_{n+1}} 1 - \frac{i}{\hbar} \left(\frac{p^2}{2m} + V\right) \Delta t +
        \mathcal{O}(\Delta t^2) \Ket{x_n} \text{ Stone's theorem}\\
    &\approx
        \Bra{x_{n+1}} e^{-\frac{i}{\hbar} \frac{p^2}{2m} \Delta t}
        e^{-\frac{i}{\hbar} V \Delta t} \Ket{x_n} \text{ Trodder product theorem} \\
    &= \int \Braket{x_{n+1} | p} \Braket{p | x_n}
        e^{-\frac{i}{\hbar} \frac{p^2}{2m} \Delta t}
        e^{-\frac{i}{\hbar} V \left(\frac{x_{n+1} + x_n}{2}\right) \Delta t}
        dp \\
    &=  e^{-\frac{i}{\hbar} V \left(\frac{x_{n+1} + x_n}{2}\right) \Delta t}
        \sqrt{\frac{m}{2 \pi i \hbar \Delta t}} \exp \left[
            \frac{i}{\hbar} \frac{m}{2} 
            \frac{\left(x_{n+1} - x_n\right)^2}{\left(\Delta t\right)^2}
        \right] \Delta t
\end{align*}
This gives a series of individual small propogators which add up to the exponential form
given at the very beginning of lecture. This derivation also gives what the measure has to be
$$
    \mathcal{D}\left[x\right] = \prod_{i=1}^{N} 
    \sqrt{\frac{m}{2 \pi i \hbar \Delta t}} dx_i
$$
There is another factor here though that he mentioned but didn't fully describe. It comes from
there being $N+1$ total propagators.

\subsection{Free particle example}
For a free particle $H = \frac{p^2}{2m}$, and we know that the form of the propagator is
$K(x, t; y, t_0) = \sqrt{\frac{m}{2 \pi i \hbar (t - t_0)}} 
e^{\frac{i}{\hbar} \frac{m \left(x-y\right)^2}{t - t_0}}
$
We can to get this propagator by integrating multiple smaller propagators like in the
Feynman path integral formalism.
\begin{align*}
    K
    &= \lim_{N \to \infty}
        \left(\frac{m}{2 \pi i \hbar \Delta t}\right)^{\frac{N+1}{2}}
        \int dx_1 dx_2 \dots dx_n
        e^{- \frac{1}{2} \frac{m}{i\hbar \Delta t}
            \sum_{n=0}^{N} \left(x_{n+1} - x_n\right)^2} \\
\end{align*}
Although this seems nasty we can use the following formulat
$$
    \frac{1}{\sqrt{2 \pi \sigma^2}} \int_{-\infty}^{\infty}
    e^{- \frac{\left(x-z\right)^2}{2 \sigma^2} - \frac{\left(z-y\right)^2}{2n \sigma^2}}
    dz = 
    e^{- \frac{\left(x-y\right)^2}{2(n + 1) \sigma^2}} \sqrt{\frac{n}{n+1}}
$$
This allows us to quickly combine the variance of all of the time steps. We will ultimately
end up with the following
$$
    K = 
        \left(\frac{m}{2 \pi i \hbar (t - t_0)}\right)^{\frac{1}{2}}
        e^{-\frac{1}{2} \frac{m}{i\hbar (N+1) \Delta t} \left(x-y\right)^2}
$$
So far we've calculated this propagator three different ways. Spectral decomposition,
Gaussian integrals, and time discretation. We can also do the same thing for other cases
but can be quite cumbersom. However, we can also apply a {\color{red} semi-classical approximation}
where we assume $\hbar$ is small relative to some order parameter. This would correspond to
a type of WKB approximation

\subsection{Semiclassical Approximation}
Say we want to approximate
\begin{align*}
    \int_{\mathbb{R}} e^{\frac{i}{\hbar} f(z)} dz \rightarrow \\
    & \sum_{\text{critical}} e^{\frac{i}{\hbar} f(z_0)}
        \int_{\mathbb{R}} e^{\frac{i}{\hbar} f'(z_0) (z - z_0) +
            \frac{i}{\hbar} \frac{1}{2} (z - z_0) f''(z_0) (z - z_0)} dz \\
    &= e^{\frac{i}{\hbar} f(z_0)}
        \sqrt{\frac{2 \pi i \hbar }{f''(z_0)}} \\
    &= e^{\frac{i}{\hbar} f(z_0)}
        e^{i \nu \frac{\pi}{4}}
        \sqrt{\frac{2 \pi \hbar}{\abs*{f''(z_0)}}}, \; \nu = \text{sgn} \left(f''(z_0)\right)
\end{align*}
This is under the case that $\hbar \to 0$. At a critical point the phases of the
individual values will constructively add together. We will want to then take this
approximate the action semi-classically. Meaning
$S(x) = S(x_{cl} + \delta x) = S(x_{cl}) + \delta S(x_{cl}) \delta x + 
\frac{1}{2} \delta^2 S \delta x^2$
It's possible for the second order paths to have positive, negative, or zero eigen values
which will cause an integration over all paths which is divergent. This is what gives rise to
regularization by controlling the overall "strength" of this second order term. Showing the
specific form of this term takes quite a bit of time so we'll skip around a bit. Let's
focus first on how the first approximation occurs
\begin{align*}
    S(x_{cl} + \delta x)
    &= \int_{t_0}^{t} \left[
        \frac{1}{2} m  \left(\dot{x}_c + \dot{\delta x}\right)^2 -
        V (x_c + \delta x)
        \right] \\
    &= S(x_{cl}) + \int_{t_0}^{t} \left[m \dot{x}_c \delta x - V'(x_c) \delta x\right] dt +
    \\&\hspace{1cm}
        \int_{t_0}^{t} \left(\frac{1}{2} m \left(\delta x\right)^2 - 
        \frac{1}{2} V''(x_c) \delta x^2\right) dt \\
    &= S(x_{cl}) + \int_{t_0}^{t}
        \delta x \left(-\frac{1}{2} m \diff[2]{}{t} - \frac{1}{2} V''(x_c)\right) \delta x
\end{align*}
This last term is the curvature of the setup and represents the Wick rotated time due to the
negative sign on everything. We can define the operator
$$
    -B = m \diff[2]{}{t} + V''(x_c)
$$
We only need to worry about the negative eigenvalues of $B$ which we will represent as
$\nu$. This finally gives rise to the Van Vlack formula which is a semi-classical approximation
$$
    K(x, t; y, t_0) \approx \frac{1}{\sqrt{2 \pi i \hbar}}
        e^{- \frac{i \pi}{2} \nu}
        \abs*{\diffp{S}{{x}{y}}}^{\frac{1}{2}}
        \exp \left(\frac{i}{\hbar} S(x, t; y, t_0)\right)
$$
where $S$ in this is the classical action. This can be used to simplify quite a few
problems under the right conditions. This formula is useful if you are able to approximate
the leading order term in a sensible way. 

The main reason to write down the Feynman path formalism is to study QFT and to understand
how particles interact with electromagnetic fields.

\section{E\&M Hamiltonian}
Let's start with a Gauge Transformation. There are many different meanings behind this term
to start it can mean "to set the initial energy" or the reference energy. In what sense
is this true? Recall from CM Hamiltons equations in phase space are
$\dot{z} = J \nabla_z H$, where $z^T = \left(\vec{p}, \vec{q}\right)$. This
definition is invariant under the shift $H \rightarrow H + H_0$ as taking the gradient
will nulify the constant. This results in the idea that
{\color{red} dynamics are invariant under an energy shift gauge transformation}.

In E\&M there is also a different degree of freedom (generated by a $U(1)$ invariance)
which allows you to choose how to represent Maxwell's equations. We can work through
this as follows using the Maxwell equations,
\begin{itemize}
    \item $\vec{E} = - \nabla \phi - \frac{1}{c} \diffp{\vec{A}}{t}$
    \item $\vec{B} = \vec{\nabla} \times \vec{A}$
\end{itemize}
Which are invariant under the following transformations,
$\phi \rightarrow \phi - \frac{1}{c} \diffp{\Lambda}{t}$ and
$\vec{A} \rightarrow \vec{A} + \nabla \Lambda$. We can do this in a covariant
way though which will help for QFT. Let's do this as follows
\begin{align*}
    \partial_{\mu} = \left(\frac{1}{c} \diffp{}{t}, \vec{\nabla}\right) &
    \partial^{\mu} = \left(\frac{1}{c} \diffp{}{t}, -\vec{\nabla}\right) \\
    A_{\mu} = \left(\phi, -\vec{A}\right) &
    A^{\mu} = \left(\phi, \vec{A}\right)
\end{align*}
Under this description we can write the gauge transformations as
$A'_{\mu} = A_{\mu} + \partial_{\mu} \Lambda$ and most of the
Maxwell formalism as $F_{\mu \nu} = \partial_{\mu} A_{\nu} - \partial_{\nu} A_{\mu}$.
There are two different common gauges.
$\vec{\nabla} \cdot \vec{A} = 0$ which is the Coulomb gauge and
$\partial_{\mu} A^{\mu} = 0$ which is the Lorentz gauge
