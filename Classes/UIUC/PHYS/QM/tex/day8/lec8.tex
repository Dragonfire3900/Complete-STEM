\section{Coherent States}
These are the QM analogue of classical motion in the phase space for the SHO.
The expectation of the position and momentum operators reproduce classical
behavior. $\Ket{n}$ SHO when quantized is equivalent to the number of photon states.
This coherent state has Poisson distribution with 
$\mathbb{P}(n) = \frac{e^{\lambda}}{n!} \lambda^n$

First quantiztion is turning things into operators. The second quantization is turning
the wave function into operators. This will give quantization for QFT.

In order to generate classical motion we need to kick the SHO in both
position and momentum space. We can use BCH theorem to combine the exponentiation
of the "kick" operator. This helps build up the $\mathcal{D}(\alpha)$ which is what the
HW is on.
\begin{align*}
    e^{\frac{i}{\hbar} p_0 \hat{x}} e^{-\frac{i}{\hbar} \hat{p}x_0}
    \Ket{0}
    &=
\end{align*}
We can define the displacement operator using this format
$$
    \mathcal{D}(\alpha) = \exp \left(
        \frac{i}{\hbar} p_0 \hat{x} - \frac{i}{\hbar} \hat{p} x_0
    \right)
$$
We can write this in terms of the creation and annihilation operators using
\begin{align*}
    \hat{x} = \gamma \left(a^\dagger + a\right), & \gamma = \sqrt{\frac{\hbar}{2m \omega}} \\
    \hat{p} = i \delta (a^\dagger - a), & \delta = \sqrt{\frac{\hbar \omega m}{2}}
\end{align*}
This gives the definitions for the annihilation and creation operators.
We can also show that the coherent state is defined using the displacement operator
$$
    \Ket{\alpha} = \mathcal{D}(\alpha) \Ket{0}
$$
We can also conclude is that the coherent state is an eigenstate of the annihilation operator
We can also conclude the expectations
\begin{align}
    \Bra{\alpha}x\Ket{\alpha} = x_0 = 2 \gamma \Re(\alpha) \\
    \Bra{\alpha}p\Ket{\alpha} = p_0 = 2;d \Im(\alpha) \\
    \Bra{\alpha}N\Ket{\alpha} = \abs*{\alpha}^2 \\
    \Braket{n | \alpha} = \left<N\right>_{\alpha}^n e^{-\left<N\right>_{\alpha}} / n! \\
\end{align}
This is in the Schrodinger picture but we can also look at the Heisenberg picture
\begin{align}
    \Bra{\alpha}x^H(t)\Ket{\alpha} = x_0 \cos \left(wt\right) + 
        \frac{p_0}{m \omega} \sin \left(\omega t\right) \\
    \Bra{\alpha} p^H(t) \Ket{\alpha} = \text{ skipped in notes}
\end{align}
In addition coherent states satisfy the minimum uncertainty principle. This comes from
translating a Gaussian still being a Gaussian.
{\color{green} We can also look at the Strum-Louiville formalism in lecture notes}

\section{WKB Approximation}
The overall idea is to linearize the potential at the crossing points at the crossing
point between two sides of the energy. This is best shown by the following linear problem
$$
    \left(- \frac{\hbar^2}{2m}\diff[2]{}{x} + k \abs*{x}\right)\psi = E \psi,
    \; k>0
$$
In general we expect that the wave function will match the parity of the Hamiltonian.
\subsection{Airy Functions}
Can build up the Airy equation by demanding that the coefficients of the two terms
are equal to one another (after shifting the linear term).
$$
    \diff[2]{u(z)}{z} = z u(z)
$$
This is very similar to SHO/exponential differential equations. It just depends on the value
of $z$. This differential equation has two different indept solutions. $Ai(z)$ which decays
as $z \to \infty$ while $B(z)$ blows up. This means that $Ai(z)$ is good for studying 
bound state particles like within the WKB approximation.

We can look at what kinds of solutions make sense based on the extremas of the Airy
functions and zeros. With the derivative solutions being the even solutions while the
odd being the function being zero.

\subsection{Asymptotic approximations}
What we care about are asymptotic approximations not just Taylor series. In particular
we need to define where the approximation is valid. In particular we have the following
definition. These expansions are not unique as we can add a small function and the ratio
will limit to $1$. In addition, asymptotic series do not always get better when adding
additional terms. There is a limit to how many terms will help. In general when you add
too many terms you will end up with a divergence.

A good example of asymptotic approximation is Stirling's Formula. For the Airy functions
there are also various Asymptotic expansions which can be incredibly helpful

\subsection{WKB Approximation}
Stands for the names of three different people. It is a general framework within mathematics
instead of just a QM approximation. It is a semi-classical approximation of 
eigenvalues and eigenvectors of the Hamiltonian. It has an assumption that he Hamiltonian
is slowly varying.

Typically is very useful in 1D. In general it's valid when the potential $V(x)$ is slowly
varying compared to the de Broglie wavelength of the particle.

