\section{Review of Lecture 18}
Let $\mathcal{H}_A$ and $\mathcal{H}_B$ be Hilbert spaces with 
an o.n. bases $\left\{\phi_n\right\}$ and $\left\{\eta_m\right\}$, respectively.
Then for any $\Ket{\psi} \in \mathcal{H}_A \otimes \mathcal{H}_B$,
$$
    \Ket{\psi} =
    \sum_{n,m} M_{nm} \Ket{\phi_n} \otimes \Ket{\eta_m}
$$
The fundamental question we have is: "Are these states entangled?"
which is the main thing we answered last time. Mainly through the following
theorem
\begin{theorem}[Entangled States]
\label{entanglesStates}
$\Ket{\psi}$ is entangled iff
$$
    S (\rho_A) = S(\rho_B) \ne 0
$$
Under singular value decomposition with
$S(\rho) = \sum_{k=1}^{N} - \sigma_k^2 \log \sigma_k^2$
\end{theorem}
One important thing to realize is that SVD representations are the same
under unitary transformations. Thus, the singular values are invariant
but the left and right decompositions can be transformed by a unitary operator
and be the same. As a note the infinite dimensional version of SVD
is {\color{red} Schmidt decomposition}. 

\subsection{Example: Partial Trace}
Let's start with the following Bell state
$\Ket{\psi} = \frac{1}{\sqrt{2}} \left(\Ket{+-} - \Ket{-+}\right)$.
First we need to compute the overall desity matrix,
$\rho_{AB} = \Ket{\psi}\Bra{\psi} = \frac{1}{2} 
\left(\Ket{+-} - \Ket{-+}\right)\left(\Bra{+-} - \Bra{-+}\right)$
We can then find the partial trace of $A$ to be
$\rho_B = \tr_A (\rho_{AB}) = \frac{1}{2} \left(\Ket{+}\Bra{+} + \Ket{-}\Bra{-}\right)$

With this we've wrapped up the density matrices and spins

\section{Adding Angular momentum}
The essential idea we are wanting to address is how the tensor product of two
different sping states can be decomposed into the direct sum of additional states
\begin{theorem}[Angular Momentum Decomposition]
\label{angDecomp}
Say we had two different spin states $V_{j_1}$ and $V_{j_2}$. The theorem is
$$
V_{j_1} \otimes V_{j_2} =
    V_{j_1 + j_2} \oplus V_{j_1 + j_2 - 1} \oplus \dots \oplus
    V_{\abs{j_1 - j_2}}
$$
\end{theorem}
This theorem will give us the Clebsh-Gordan coefficients. This requires a process to
show that you can get individual points in the tensor product space of $V_{j_1}$ and $V_{j_2}$.
We can construct individual diagonal elements by applying $J_{\pm}$ to the heighest weight
state. However after applying this once there are is another valid state which is orthogonal
to the result of $J_{-} V_{j_1 + j_2}$. This turns out to be the $V_{j_1 + j_2 - 1}$ 
space. We can see this by analyzing the orthogonality between this result and the other
parts of the space. We can then further construct the additional parts of the space by
looking at the orthogonality.

In order to do this we also need a convention in how we write our results and we
have the following (based on an old committee at some point)
\begin{enumerate}
    \item $J_{\pm}$ must have non-negative matrix elements. Which gives \newline
        $J_{\pm} \Ket{j m} = \hbar \sqrt{\left((j\mp m) (j\pm m - 1)\right)}
        \Ket{j m\pm 1}$
    \item $\Braket{J=j, M=j | j_1, j_2; m_1 = j_1, m_2 = j - j_1} > 0$. In other
        words if $m_1$ is the top value this term must be positive.
\end{enumerate}

\section{Clebsh Gordon Coefficients}
\subsection{Example Compute the Clebsh Gordon Coefficients}
There's a lot that I didn't really get to record properly but we can write out the individual
terms present in \ref{angDecomp}

\section{Adding Orbital and Spin}
We can write down $J = L + S$ which ends up adding the two forms of 
angular momentum. In this case we can represent the entire vector space they build
with the tensor product between $V_L$ and $V_S$. In the case that $S = 1/2$
we can then use \ref{angDecom} to uniquely construct the state. In particular
starting with the top state we can define
$$
    \Ket{J = l + \frac{1}{2}; M = l + \frac{1}{2}} = \Ket{l, \frac{1}{2}; l, \frac{1}{2}}
$$
Using this and the previously established conventions we can derive the general states to be
the following (which can also be represented 
using spinor notation which is also available in Sakari)
\begin{enumerate}
    \item $\Ket{J = l + \frac{1}{2}; m}$
    \item $\Ket{J = l - \frac{1}{2}; m}$
\end{enumerate}
{\color{red}Both of these are present in Sakari}. We are working towards establishing
the Wignar-Ekhardt theorem.

\section{Operator transformation under rotation}
A little bit ago we showed that the spin rotates like a vector. This also
applies for the general angular momentum components. In other words
$$
\mathcal{D}^\dagger(R) \vec{J} \mathcal{D}(R) = R \vec{J}
$$
We can also look at any general vectoral quantity and see how this
must transform. To be more specific let's say we have $\vec{V}$ with elements
$v_i: \mathcal{H} \rightarrow \mathcal{H}$, by the definition
this vector must satisfy
$\Bra{\psi} \vec{V} \Ket{\psi} \rightarrow R \Bra{\psi} \vec{V} \Ket{\psi}$.
This will result in the following operator identity
$$
\mathcal{D}^\dagger (R) \vec{V} \mathcal{D} = R \vec{V}
$$
Recalling that $\mathcal{D}(R)$ is found via the exponential of various generators
we can look at how $\vec{V}$ changes under rotation by expanding either side
in the small angle approximation. We know from problem set \#1 how vectors should
transform under rotation. We then can match the various terms together in order to find
the commutation relation of
$$
    \left[J_a, V_b\right] = i \hbar \epsilon_{abc} V_c
$$
We can then start looking at higher order rotations using the spherical harmonics as
as basis. In particular we can look at how the spherical harmonics rotate (of $l>1$)
in order to describe how higher order tensors will rotate. In particular we must
demand that
$$
\mathcal{D}^\dagger (R) Y_l^m (\vec{V}) \mathcal{D}(R) = Y_l^m (R \vec{V})
$$
which matches how they should rotate.
